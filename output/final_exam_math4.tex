\documentclass[11pt]{article}%
\usepackage[T1]{fontenc}%
\usepackage[utf8]{inputenc}%
\usepackage{lmodern}%
\usepackage{textcomp}%
\usepackage{lastpage}%
\usepackage[margin=1in]{geometry}%
\usepackage{amsmath}%
\usepackage{amssymb}%
\usepackage{titlesec}%
\usepackage{fancyhdr}%
%
%
%
\begin{document}%
\normalsize%
\begin{titlepage}%
\centering%
\vspace*{1cm}%
{\Huge \textbf{ONTARIO TECH UNIVERSITY} \par}%
\vspace{1.5cm}%
{\Large \textbf{Final Exam: Introductory Calculus Lecture 1} \par}%
\vspace{0.5cm}%
{\large \today \par}%
\vspace{2cm}%
\textbf{INSTRUCTIONS TO CANDIDATES} \par%
\vspace{0.5cm}%
\textit{(Formal and Rigorous. Instructions will require students to show all steps clearly, use precise mathematical notation, and explicitly state any formulas (like the Integration by Parts formula) used in their solutions. Tone) Closed-Book, Procedural and Application-Based Examination. The exam will consist primarily of multi-step calculation problems requiring the application of specific integration and differential equation solving techniques. Exam. Please answer all questions.}%
\vspace{3cm}%
\textbf{DO NOT OPEN THIS BOOKLET UNTIL TOLD TO DO SO} \par%
\vfill%
{\large Total Points: 100 \par}%
\end{titlepage}%
\newpage%
\section*{Questions}%
\label{sec:Questions}%
\begin{enumerate}%
\item%
What defines the "order" of an Ordinary Differential Equation (ODE)? (3 points)%
\begin{itemize}%
\item%
The number of variables involved.%
\item%
The highest power of the independent variable.%
\item%
The order of the highest derivative that occurs.%
\item%
The number of terms in the equation.%
\end{itemize}%
\vspace{0.5cm}%
\item%
State the correct formula for Integration by Parts, where $u$ and $v$ are differentiable functions of $x$. (3 points)%
\begin{itemize}%
\item%
$$\int u dv = uv - \int v du$$%
\item%
$$\int u dv = vu + \int v du$$%
\item%
$$\int u dv = u'v - \int uv' du$$%
\item%
$$\int u dv = \frac{u}{v} - \int \frac{v}{u} dv$$%
\end{itemize}%
\vspace{0.5cm}%
\item%
A first-order differential equation is classified as separable if it can be written in which of the following forms? (3 points)%
\begin{itemize}%
\item%
$$\frac{dy}{dx} = f(x) + g(y)$$%
\item%
$$\frac{dy}{dx} = a(x)b(y)$$%
\item%
$$\frac{dy}{dx} + P(x)y = Q(x)$$%
\item%
$$\frac{dy}{dx} = \frac{y}{x}$$%
\end{itemize}%
\vspace{0.5cm}%
\item%
According to Kirchhoff's laws, the differential equation modeling the charge $Q(t)$ in a series RLC circuit with voltage source $V(t)$ is given by: (3 points)%
\begin{itemize}%
\item%
$$R\frac{dQ}{dt} + \frac{1}{C}Q = V(t)$$%
\item%
$$L\frac{d^2Q}{dt^2} + R\frac{dQ}{dt} + \frac{1}{C}Q = V(t)$$%
\item%
$$L\frac{dI}{dt} + R I = V(t)$$%
\item%
$$\frac{d^2Q}{dt^2} + \frac{R}{L}\frac{dQ}{dt} + \frac{1}{LC}Q = 0$$%
\end{itemize}%
\vspace{0.5cm}%
\item%
When applying Newton's Second Law ($F=ma$) to model the motion of an object, the resulting differential equation is typically of what order, assuming displacement $r(t)$ is the dependent variable? (3 points)%
\begin{itemize}%
\item%
Zero order%
\item%
First order%
\item%
Second order%
\item%
Third order%
\end{itemize}%
\vspace{0.5cm}%
\item%
The rate of radioactive decay is proportional to the remaining number of atoms $N$. Which differential equation correctly models this phenomenon? ($k$ is a positive constant) (3 points)%
\begin{itemize}%
\item%
$$\frac{dN}{dt} = k t$$%
\item%
$$\frac{dN}{dt} = k N^2$$%
\item%
$$\frac{dN}{dt} = -k N$$%
\item%
$$\frac{dN}{dt} = k N + C$$%
\end{itemize}%
\vspace{0.5cm}%
\item%
Evaluate the indefinite integral $$\int x e^{3x} dx$$. (3 points)%
\begin{itemize}%
\item%
$$\frac{1}{3}x e^{3x} - \frac{1}{9}e^{3x} + C$$%
\item%
$$3x e^{3x} - 9e^{3x} + C$$%
\item%
$$\frac{1}{3}x e^{3x} + \frac{1}{9}e^{3x} + C$$%
\item%
$$x e^{3x} - e^{3x} + C$$%
\end{itemize}%
\vspace{0.5cm}%
\item%
Evaluate the indefinite integral $$\int x^2 \ln(x) dx$$. (3 points)%
\begin{itemize}%
\item%
$$\frac{x^3}{3} \ln(x) - \frac{x^3}{9} + C$$%
\item%
$$\frac{x^3}{3} \ln(x) + \frac{x^3}{9} + C$$%
\item%
$$x^3 \ln(x) - x^2 + C$$%
\item%
$$\frac{x^3}{3} \ln(x) - \frac{x^2}{3} + C$$%
\end{itemize}%
\vspace{0.5cm}%
\item%
Evaluate the indefinite integral $$\int x \cos(2x) dx$$. (3 points)%
\begin{itemize}%
\item%
$$\frac{1}{2}x \sin(2x) + \frac{1}{4}\cos(2x) + C$$%
\item%
$$x \sin(2x) - \cos(2x) + C$$%
\item%
$$\frac{1}{2}x \sin(2x) - \frac{1}{4}\cos(2x) + C$$%
\item%
$$2x \sin(2x) + 4\cos(2x) + C$$%
\end{itemize}%
\vspace{0.5cm}%
\item%
Given the reduction formula for $I_n = \int \cos^n(x) dx$ is $$I_n = \frac{1}{n}\cos^{n-1}(x)\sin(x) + \frac{n-1}{n}I_{n-2}$$. If $I_4$ is calculated, what coefficient multiplies $I_2$? (3 points)%
\begin{itemize}%
\item%
$$\frac{3}{4}$$%
\item%
$$\frac{1}{4}$$%
\item%
$$\frac{1}{2}$$%
\item%
$$\frac{4}{3}$$%
\end{itemize}%
\vspace{0.5cm}%
\item%
Which integral typically requires applying the Integration by Parts technique twice (or using a cyclical method) for full solution? (3 points)%
\begin{itemize}%
\item%
$$\int \ln(x) dx$$%
\item%
$$\int x e^x dx$$%
\item%
$$\int e^x \sin(x) dx$$%
\item%
$$\int \frac{1}{x} \ln(x) dx$$%
\end{itemize}%
\vspace{0.5cm}%
\item%
Find the general solution to the differential equation $$\frac{dy}{dx} = 4x^3 y$$. (3 points)%
\begin{itemize}%
\item%
$$y = C e^{x^4}$$%
\item%
$$y = x^4 + C$$%
\item%
$$\ln|y| = 4x^3 + C$$%
\item%
$$y = C x^4$$%
\end{itemize}%
\vspace{0.5cm}%
\item%
Solve the initial value problem $$\frac{dy}{dx} = \frac{x}{y}$$ with the condition $y(0) = 2$. (3 points)%
\begin{itemize}%
\item%
$$y^2 = x^2 + 4$$%
\item%
$$y = \sqrt{x^2 + 2}$$%
\item%
$$y^2 = x^2 + 2$$%
\item%
$$y = 2e^{x^2/2}$$%
\end{itemize}%
\vspace{0.5cm}%
\item%
Find the general solution to the differential equation $$\frac{dy}{dx} = \frac{\sin(x)}{\cos(y)}$$. (3 points)%
\begin{itemize}%
\item%
$$\sin(y) = -\cos(x) + C$$%
\item%
$$\cos(y) = \sin(x) + C$$%
\item%
$$\sin(y) = \cos(x) + C$$%
\item%
$$\cos(y) = -\sin(x) + C$$%
\end{itemize}%
\vspace{0.5cm}%
\item%
Solve the separable equation $$(1+x^2) \frac{dy}{dx} = y$$. (3 points)%
\begin{itemize}%
\item%
$$y = C \arctan(x)$$%
\item%
$$y = C e^{\arctan(x)}$$%
\item%
$$y = C \ln(1+x^2)$$%
\item%
$$y = C (1+x^2)$$%
\end{itemize}%
\vspace{0.5cm}%
\item%
Evaluate the indefinite integral using integration by parts. Show all steps clearly, including both applications of the formula: $$\int x^2 e^{3x} dx$$ (5 points)%
\vspace{4cm}%
\vspace{0.5cm}%
\item%
Find the general solution to the first-order separable differential equation. Show the separation and integration steps clearly: $$\frac{dy}{dx} = \frac{x \cos(x^2)}{y^3}$$ (5 points)%
\vspace{4cm}%
\vspace{0.5cm}%
\item%
A series RLC circuit has resistance $R$, inductance $L$, capacitance $C$, and is driven by an external voltage source $V(t)$.

1. Write down the second-order ordinary differential equation that models the charge $Q(t)$ on the capacitor, based on Kirchhoff's laws.
2. State the order of this ODE.
3. If the voltage source is removed ($V(t)=0$), classify the resulting ODE as either homogeneous or non-homogeneous. (5 points)%
\vspace{4cm}%
\vspace{0.5cm}%
\item%
The reduction formula for $I_n = \int \cos^n(x) dx$ is often derived using integration by parts. Show the application of integration by parts on $I_n = \int \cos^{n-1}(x) \cos(x) dx$ that leads to the recursive integral term. Clearly define $u$ and $dv$ and show the resulting expression $I_n = uv - \int v du$. (5 points)%
\vspace{4cm}%
\vspace{0.5cm}%
\item%
1. Define the 'Order' of a differential equation.
2. Define the term 'Ordinary Differential Equation (ODE)'.
3. Classify the following equation by its Order and Linearity: $$\frac{d^3y}{dx^3} + 5x^2 \frac{dy}{dx} + y = \ln(x)$$ (5 points)%
\vspace{4cm}%
\vspace{0.5cm}%
\item%
Part A (8 points): Evaluate the definite integral:
$$\int_{0}^{\pi/2} x^2 \cos(2x) dx$$
Show all steps clearly, including the choice of $u$ and $dv$ for each application of the Integration by Parts formula.

Part B (2 points): State the general formula for Integration by Parts using $u$ and $v$ and define the order of an Ordinary Differential Equation (ODE). (10 points)%
\vspace{8cm}%
\vspace{0.5cm}%
\item%
Consider the first-order differential equation:
$$\frac{dy}{dx} = \frac{x e^{2x}}{y \sqrt{1+y^2}}$$
Part A (8 points): Find the general implicit solution to this separable differential equation. Show all steps of separation and integration clearly.

Part B (2 points): If the equation were modified to $\frac{dy}{dx} = x e^{2x} + y \sqrt{1+y^2}$, explain why the method used in Part A (Separation of Variables) would fail. (10 points)%
\vspace{8cm}%
\vspace{0.5cm}%
\item%
Part A (3 points): Consider the RLC circuit differential equation:
$$L \frac{d^2Q}{dt^2} + R \frac{dQ}{dt} + \frac{1}{C} Q = V(t)$$
Identify the order and linearity (linear or non-linear) of this Ordinary Differential Equation (ODE). Justify your answers based on the definitions of ODE properties.

Part B (7 points): The rate of decay of a radioactive substance, $N(t)$, is proportional to the amount of the substance present at time $t$. This phenomenon is described by a first-order separable ODE.

1. Set up the differential equation model using a proportionality constant $k$ (where $k > 0$). (2 points)
2. Solve the differential equation to find the general expression for $N(t)$. (5 points) (10 points)%
\vspace{8cm}%
\vspace{0.5cm}%
\end{enumerate}

%
\newpage%
\section*{Solution Key & Grading Rubric}%
\label{sec:SolutionKeyGradingRubric}%
\textbf{Confidential - Instructor Use Only}%
\vspace{0.5cm}%
\begin{enumerate}%
\item%
\textbf{M-1}%
 \ \ \textbf{Model Answer:} The order of the highest derivative that occurs.%
 \\ \textit{Grading Notes:} 3 pts for correctly identifying the definition of the order of an ODE (Knowledge Recall).%
\vspace{0.3cm}%
\item%
\textbf{M-2}%
 \ \ \textbf{Model Answer:} $$\int u dv = uv - \int v du$$%
 \\ \textit{Grading Notes:} 3 pts for exact recall of the standard Integration by Parts formula (Knowledge Recall).%
\vspace{0.3cm}%
\item%
\textbf{M-3}%
 \ \ \textbf{Model Answer:} $$\frac{dy}{dx} = a(x)b(y)$$%
 \\ \textit{Grading Notes:} 3 pts for identifying the standard multiplicative form required for separation of variables (Knowledge Recall).%
\vspace{0.3cm}%
\item%
\textbf{M-4}%
 \ \ \textbf{Model Answer:} $$L\frac{d^2Q}{dt^2} + R\frac{dQ}{dt} + \frac{1}{C}Q = V(t)$$%
 \\ \textit{Grading Notes:} 3 pts for correctly identifying the second-order non-homogeneous ODE for the RLC circuit charge $Q$. (1 pt for recognizing the second-order structure, 2 pts for correct coefficients/terms) (Conceptual Application).%
\vspace{0.3cm}%
\item%
\textbf{M-5}%
 \ \ \textbf{Model Answer:} Second order%
 \\ \textit{Grading Notes:} 3 pts for recognizing that acceleration ($a$) is the second derivative of displacement ($a = d^2r/dt^2$), resulting in a second-order ODE (Conceptual Application).%
\vspace{0.3cm}%
\item%
\textbf{M-6}%
 \ \ \textbf{Model Answer:} $$\frac{dN}{dt} = -k N$$%
 \\ \textit{Grading Notes:} 3 pts for correctly setting up the proportional decay model, including the negative sign to indicate decrease (Conceptual Application).%
\vspace{0.3cm}%
\item%
\textbf{M-7}%
 \ \ \textbf{Model Answer:} $$\frac{1}{3}x e^{3x} - \frac{1}{9}e^{3x} + C$$%
 \\ \textit{Grading Notes:} 1 pt for correct setup ($u=x, dv=e^{3x}dx$). 1 pt for correct application of the integration by parts formula. 1 pt for the final correct integration (Procedural Fluency).%
\vspace{0.3cm}%
\item%
\textbf{M-8}%
 \ \ \textbf{Model Answer:} $$\frac{x^3}{3} \ln(x) - \frac{x^3}{9} + C$$%
 \\ \textit{Grading Notes:} 1 pt for correct setup ($u=\ln(x), dv=x^2 dx$). 1 pt for correct application of the formula. 1 pt for correctly integrating the remaining polynomial term (Procedural Fluency).%
\vspace{0.3cm}%
\item%
\textbf{M-9}%
 \ \ \textbf{Model Answer:} $$\frac{1}{2}x \sin(2x) + \frac{1}{4}\cos(2x) + C$$%
 \\ \textit{Grading Notes:} 1 pt for correct setup ($u=x, dv=\cos(2x)dx$). 1 pt for correct application of the formula. 1 pt for correct integration of $\sin(2x)$ (Procedural Fluency).%
\vspace{0.3cm}%
\item%
\textbf{M-10}%
 \ \ \textbf{Model Answer:} $$\frac{3}{4}$$%
 \\ \textit{Grading Notes:} 3 pts for correctly substituting $n=4$ into the reduction formula and identifying the coefficient $\frac{n-1}{n} = \frac{3}{4}$ (Procedural Fluency).%
\vspace{0.3cm}%
\item%
\textbf{M-11}%
 \ \ \textbf{Model Answer:} $$\int e^x \sin(x) dx$$%
 \\ \textit{Grading Notes:} 3 pts for recognizing that the product of an exponential and a trigonometric function often requires two applications of integration by parts to return to the original integral (Procedural Fluency).%
\vspace{0.3cm}%
\item%
\textbf{M-12}%
 \ \ \textbf{Model Answer:} $$y = C e^{x^4}$$%
 \\ \textit{Grading Notes:} 1 pt for correct separation $$\frac{dy}{y} = 4x^3 dx$$. 1 pt for correct integration $$\ln|y| = x^4 + K$$. 1 pt for solving for $y$ and handling the constant (Problem Solving).%
\vspace{0.3cm}%
\item%
\textbf{M-13}%
 \ \ \textbf{Model Answer:} $$y^2 = x^2 + 4$$%
 \\ \textit{Grading Notes:} 1 pt for correct integration leading to $y^2/2 = x^2/2 + K$. 1 pt for correctly applying the initial condition $y(0)=2$ to find $C=4$. 1 pt for the final specific solution (Problem Solving).%
\vspace{0.3cm}%
\item%
\textbf{M-14}%
 \ \ \textbf{Model Answer:} $$\sin(y) = -\cos(x) + C$$%
 \\ \textit{Grading Notes:} 1 pt for correct separation $$\cos(y) dy = \sin(x) dx$$. 1 pt for integrating $\cos(y)$ correctly. 1 pt for integrating $\sin(x)$ correctly (Problem Solving).%
\vspace{0.3cm}%
\item%
\textbf{M-15}%
 \ \ \textbf{Model Answer:} $$y = C e^{\arctan(x)}$$%
 \\ \textit{Grading Notes:} 1 pt for correct separation $$\frac{dy}{y} = \frac{dx}{1+x^2}$$. 1 pt for recognizing the integral of the RHS is $\arctan(x)$. 1 pt for the final general solution form after exponentiation (Problem Solving).%
\vspace{0.3cm}%
\item%
\textbf{S-1}%
 \ \ \textbf{Model Answer:} This integral requires two applications of integration by parts ($\int u dv = uv - \int v du$).

**First Application:**
Let $u = x^2$ and $dv = e^{3x} dx$.
Then $du = 2x dx$ and $v = \frac{1}{3}e^{3x}$.
$$\int x^2 e^{3x} dx = \frac{1}{3}x^2 e^{3x} - \int \frac{1}{3}e^{3x} (2x) dx = \frac{1}{3}x^2 e^{3x} - \frac{2}{3}\int x e^{3x} dx$$

**Second Application (on $\int x e^{3x} dx$):**
Let $u = x$ and $dv = e^{3x} dx$.
Then $du = dx$ and $v = \frac{1}{3}e^{3x}$.
$$\int x e^{3x} dx = \frac{1}{3}x e^{3x} - \int \frac{1}{3}e^{3x} dx = \frac{1}{3}x e^{3x} - \frac{1}{9}e^{3x} + C_1$$

**Final Solution:**
Substitute the result back into the first equation:
$$\frac{1}{3}x^2 e^{3x} - \frac{2}{3} \left( \frac{1}{3}x e^{3x} - \frac{1}{9}e^{3x} \right) + C$$
$$\frac{1}{3}x^2 e^{3x} - \frac{2}{9}x e^{3x} + \frac{2}{27}e^{3x} + C$$%
 \\ \textit{Grading Notes:} 1 pt for correct setup of the first IBP. 1 pt for correctly executing the first IBP resulting in the second integral. 2 pts for correctly executing the second IBP and integrating the final exponential term. 1 pt for combining all terms correctly and including the constant of integration $C$.%
\vspace{0.3cm}%
\item%
\textbf{S-2}%
 \ \ \textbf{Model Answer:} 1. **Separate Variables:**
$$y^3 dy = x \cos(x^2) dx$$
2. **Integrate Both Sides:**
$$\int y^3 dy = \int x \cos(x^2) dx$$
3. **Integrate LHS:**
$$\int y^3 dy = \frac{y^4}{4} + C_1$$
4. **Integrate RHS (using substitution $u=x^2, du=2x dx$):**
$$\int x \cos(x^2) dx = \frac{1}{2} \int \cos(u) du = \frac{1}{2} \sin(u) + C_2 = \frac{1}{2} \sin(x^2) + C_2$$
5. **General Solution:**
$$\frac{y^4}{4} = \frac{1}{2} \sin(x^2) + C$$ (where $C = C_2 - C_1$)
(Alternatively: $y^4 = 2 \sin(x^2) + K$, where $K=4C$)%
 \\ \textit{Grading Notes:} 1 pt for correctly separating the variables. 1 pt for correctly integrating the LHS (y terms). 2 pts for correctly integrating the RHS (x terms), requiring the substitution method. 1 pt for the final general solution form including the arbitrary constant $C$.%
\vspace{0.3cm}%
\item%
\textbf{S-3}%
 \ \ \textbf{Model Answer:} 1. **RLC Circuit ODE (Charge Q(t)):**
$$L\frac{d^2Q}{dt^2} + R\frac{dQ}{dt} + \frac{1}{C}Q = V(t)$$
2. **Order:**
Second-order.
3. **Classification (if $V(t)=0$):**
Homogeneous.%
 \\ \textit{Grading Notes:} 3 pts for the correct ODE formula (1 pt for the $L$ term, 1 pt for the $R$ term, 1 pt for the $C$ term, correctly equated to $V(t)$). 1 pt for correctly stating the order (Second-order). 1 pt for correctly classifying the $V(t)=0$ case (Homogeneous).%
\vspace{0.3cm}%
\item%
\textbf{S-4}%
 \ \ \textbf{Model Answer:} We set up the integral for Integration by Parts ($\int u dv = uv - \int v du$):

1. **Define u and dv:**
$$u = \cos^{n-1}(x)$$
$$dv = \cos(x) dx$$

2. **Find du and v:**
$$du = (n-1)\cos^{n-2}(x) (-\sin(x)) dx = -(n-1)\cos^{n-2}(x)\sin(x) dx$$
$$v = \sin(x)$$

3. **Apply IBP Formula:**
$$I_n = \cos^{n-1}(x)\sin(x) - \int \sin(x) \left[ -(n-1)\cos^{n-2}(x)\sin(x) \right] dx$$

4. **Simplify the Integral Term:**
$$I_n = \cos^{n-1}(x)\sin(x) + (n-1) \int \cos^{n-2}(x)\sin^2(x) dx$$ (This step shows the successful application of IBP leading to the recursive integral term.)%
 \\ \textit{Grading Notes:} 1 pt for correctly splitting the integral into $u$ and $dv$. 2 pts for correctly calculating $du$ and $v$. 1 pt for correctly substituting $u, v, du$ into the IBP formula. 1 pt for the resulting expression showing the reduction in the power of cosine and the introduction of $\sin^2(x)$.%
\vspace{0.3cm}%
\item%
\textbf{S-5}%
 \ \ \textbf{Model Answer:} 1. **Order Definition:** The order of a differential equation is determined by the order of the highest derivative present in the equation.
2. **ODE Definition:** An Ordinary Differential Equation (ODE) is an equation involving an independent variable, a function of that variable, and only ordinary derivatives of that function (as opposed to partial derivatives).
3. **Classification:**
    *   **Order:** Third-order (due to the presence of $\frac{d^3y}{dx^3}$). 
    *   **Linearity:** Linear (because $y$ and all its derivatives appear only to the first power and are not multiplied together or involved in non-linear functions).%
 \\ \textit{Grading Notes:} 1 pt for the correct definition of Order. 1 pt for the correct definition of ODE. 1 pt for correctly identifying the Order of the example equation (Third-order). 2 pts for correctly identifying the Linearity of the example equation (Linear).%
\vspace{0.3cm}%
\item%
\textbf{P-1}%
 \ \ \textbf{Model Answer:} Part A: Evaluation of $\int_{0}^{\pi/2} x^2 \cos(2x) dx$.

We use Integration by Parts (IBP), $\int u dv = uv - \int v du$.

First Application:
Let $u = x^2$, $dv = \cos(2x) dx$.
Then $du = 2x dx$, $v = \frac{1}{2} \sin(2x)$.

$$\int x^2 \cos(2x) dx = \left[ \frac{1}{2} x^2 \sin(2x) \right]_{0}^{\pi/2} - \int_{0}^{\pi/2} \frac{1}{2} \sin(2x) (2x) dx$$
$$\int x^2 \cos(2x) dx = \left[ \frac{1}{2} (\frac{\pi}{2})^2 \sin(\pi) - 0 \right] - \int_{0}^{\pi/2} x \sin(2x) dx$$
Since $\sin(\pi) = 0$, the first term is 0.
$$I = - \int_{0}^{\pi/2} x \sin(2x) dx$$

Second Application (on $I_1 = \int x \sin(2x) dx$):
Let $u = x$, $dv = \sin(2x) dx$.
Then $du = dx$, $v = -\frac{1}{2} \cos(2x)$.

$$I_1 = \left[ -\frac{1}{2} x \cos(2x) \right]_{0}^{\pi/2} - \int_{0}^{\pi/2} -\frac{1}{2} \cos(2x) dx$$
$$I_1 = \left[ -\frac{1}{2} (\frac{\pi}{2}) \cos(\pi) - 0 \right] + \frac{1}{2} \int_{0}^{\pi/2} \cos(2x) dx$$
Since $\cos(\pi) = -1$:
$$I_1 = \frac{\pi}{4} + \frac{1}{2} \left[ \frac{1}{2} \sin(2x) \right]_{0}^{\pi/2}$$
$$I_1 = \frac{\pi}{4} + \frac{1}{4} [\sin(\pi) - \sin(0)] = \frac{\pi}{4} + 0 = \frac{\pi}{4}$$

Final Result: $I = -I_1 = -\frac{\pi}{4}$.

Part B:
1. Integration by Parts Formula: $\int u dv = uv - \int v du$.
2. Order of an ODE: The order of the highest derivative that occurs in the differential equation.%
 \\ \textit{Grading Notes:} Part A (8 points):
1. Correct setup of the first IBP ($u=x^2, dv=\cos(2x) dx$) and calculation of $v$ and $du$. (2 points)
2. Correct evaluation of the boundary terms for the first IBP (resulting in 0). (1 point)
3. Correct setup of the second IBP ($u=x, dv=\sin(2x) dx$) and calculation of $v$ and $du$. (2 points)
4. Correct integration of the remaining term $\int \cos(2x) dx$ and evaluation of its boundary terms. (2 points)
5. Correct final numerical answer, $I = -\frac{\pi}{4}$. (1 point)

Part B (2 points):
1. Correct statement of the IBP formula $\int u dv = uv - \int v du$. (1 point)
2. Correct definition of the order of an ODE. (1 point)%
\vspace{0.3cm}%
\item%
\textbf{P-2}%
 \ \ \textbf{Model Answer:} Part A: Solving the separable ODE.

1. Separation of Variables:
$$\frac{dy}{dx} = \frac{x e^{2x}}{y \sqrt{1+y^2}}$$
$$y \sqrt{1+y^2} dy = x e^{2x} dx$$

2. Integration:
$$\int y \sqrt{1+y^2} dy = \int x e^{2x} dx$$

3. Integrating the LHS (Substitution):
Let $u = 1+y^2$, so $du = 2y dy$, or $y dy = \frac{1}{2} du$.
$$\int \sqrt{u} \left( \frac{1}{2} du \right) = \frac{1}{2} \int u^{1/2} du = \frac{1}{2} \cdot \frac{2}{3} u^{3/2} + C_1$$
$$\text{LHS} = \frac{1}{3} (1+y^2)^{3/2}$$

4. Integrating the RHS (Integration by Parts):
Let $w = x$, $dz = e^{2x} dx$. Then $dw = dx$, $z = \frac{1}{2} e^{2x}$.
$$\int x e^{2x} dx = \frac{1}{2} x e^{2x} - \int \frac{1}{2} e^{2x} dx$$
$$\text{RHS} = \frac{1}{2} x e^{2x} - \frac{1}{4} e^{2x} + C_2$$

5. General Implicit Solution (Combining $C_1$ and $C_2$ into $C$):
$$\frac{1}{3} (1+y^2)^{3/2} = \frac{1}{4} e^{2x} (2x - 1) + C$$

Part B: Failure of Separation of Variables.

The modified equation is $\frac{dy}{dx} = x e^{2x} + y \sqrt{1+y^2}$. This equation is not separable because the right-hand side is a sum of a function of $x$ and a function of $y$, $f(x) + g(y)$. It cannot be factored into the required form $a(x)b(y)$ necessary for separation of variables.%
 \\ \textit{Grading Notes:} Part A (8 points):
1. Correctly separating the variables: $y \sqrt{1+y^2} dy = x e^{2x} dx$. (2 points)
2. Correctly integrating the LHS using substitution, resulting in $\frac{1}{3} (1+y^2)^{3/2}$. (3 points)
3. Correctly integrating the RHS using Integration by Parts, resulting in $\frac{1}{4} e^{2x} (2x - 1)$. (2 points)
4. Stating the final general implicit solution including the constant $C$. (1 point)

Part B (2 points):
1. Correctly identifying that the RHS is a sum $f(x) + g(y)$. (1 point)
2. Correctly explaining that the equation cannot be written in the form $a(x)b(y)$, thus preventing separation. (1 point)%
\vspace{0.3cm}%
\item%
\textbf{P-3}%
 \ \ \textbf{Model Answer:} Part A: RLC Circuit ODE Analysis.

1. Order: The highest derivative present is the second derivative, $\frac{d^2Q}{dt^2}$. Therefore, the ODE is **Second Order**.
2. Linearity: The dependent variable $Q$ and all its derivatives ($\frac{dQ}{dt}$, $\frac{d^2Q}{dt^2}$) appear only to the first power, and there are no products of $Q$ or its derivatives. The coefficients ($L, R, 1/C$) are constants (or functions of $t$ only, if $V(t)$ is the forcing function). Therefore, the ODE is **Linear**.

Part B: Radioactive Decay Model.

1. Setting up the ODE:
The rate of decay ($\frac{dN}{dt}$) is proportional to the amount present ($N$). Since it is decay, the rate must be negative.
$$\frac{dN}{dt} = -kN \quad \text{where } k > 0$$

2. Solving the ODE (Separation of Variables):
$$\frac{dN}{N} = -k dt$$
Integrate both sides:
$$\int \frac{1}{N} dN = \int -k dt$$
$$\ln|N| = -kt + C_1$$
Exponentiate both sides:
$$|N| = e^{-kt + C_1} = e^{C_1} e^{-kt}$$
Let $A = \pm e^{C_1}$ (where $A$ is an arbitrary constant representing the initial amount $N_0$):
$$N(t) = A e^{-kt}$$%
 \\ \textit{Grading Notes:} Part A (3 points):
1. Correctly identifying the order as Second Order. (1 point)
2. Correctly identifying the equation as Linear. (1 point)
3. Providing a valid justification for linearity (dependent variable and derivatives appear only to the first power, no products). (1 point)

Part B (7 points):
1. Correctly setting up the differential equation $\frac{dN}{dt} = -kN$ (or equivalent, using a negative sign to denote decay). (2 points)
2. Correctly separating the variables: $\frac{dN}{N} = -k dt$. (1 point)
3. Correctly integrating both sides: $\ln|N| = -kt + C_1$. (2 points)
4. Correctly solving for $N(t)$ to obtain the general solution $N(t) = A e^{-kt}$. (2 points)%
\vspace{0.3cm}%
\end{enumerate}

%
\end{document}