\documentclass[11pt]{article}%
\usepackage[T1]{fontenc}%
\usepackage[utf8]{inputenc}%
\usepackage{lmodern}%
\usepackage{textcomp}%
\usepackage{lastpage}%
\usepackage[margin=1in]{geometry}%
\usepackage{amsmath}%
\usepackage{amssymb}%
\usepackage{titlesec}%
\usepackage{fancyhdr}%
%
%
%
\begin{document}%
\normalsize%
\begin{titlepage}%
\centering%
\vspace*{1cm}%
{\Huge \textbf{ONTARIO TECH UNIVERSITY} \par}%
\vspace{1.5cm}%
{\Large \textbf{Final Exam: Introductory Calculus Lecture 1} \par}%
\vspace{0.5cm}%
{\large \today \par}%
\vspace{2cm}%
\textbf{INSTRUCTIONS TO CANDIDATES} \par%
\vspace{0.5cm}%
\textit{(Formal, precise, and rigorous. Instructions will demand that students 'Show all work,' 'Clearly state any substitutions used,' and 'Derive the final general solution, including the constant of integration.' Tone) Structured Problem Solving Exam. The assessment will consist primarily of multi-step calculation problems requiring the application of specific integration techniques (Integration by Parts, reduction formulas) and the solution of basic differential equations (Separable Equations). A portion of the exam will test the ability to formulate differential equations from physical descriptions (e.g., RLC circuits, mechanics). Exam. Please answer all questions.}%
\vspace{3cm}%
\textbf{DO NOT OPEN THIS BOOKLET UNTIL TOLD TO DO SO} \par%
\vfill%
{\large Total Points: 100 \par}%
\end{titlepage}%
\newpage%
\section*{Questions}%
\label{sec:Questions}%
\begin{enumerate}%
\item%
Evaluate the indefinite integral $\int x e^{3x} dx$ using integration by parts. (3 points)%
\begin{itemize}%
\item%
A. $\frac{x}{3}e^{3x} - \frac{1}{9}e^{3x} + C$%
\item%
B. $3x e^{3x} - 9e^{3x} + C$%
\item%
C. $x e^{3x} - e^{3x} + C$%
\item%
D. $\frac{x^2}{2} e^{3x} - \int \frac{x^2}{2} 3e^{3x} dx$%
\end{itemize}%
\vspace{0.5cm}%
\item%
Which choice of $u$ and $dv$ is most appropriate for evaluating $\int x^2 \ln(x) dx$ using integration by parts? (3 points)%
\begin{itemize}%
\item%
A. $u=x^2$, $dv=\ln(x)dx$%
\item%
B. $u=\ln(x)$, $dv=x^2 dx$%
\item%
C. $u=1$, $dv=x^2 \ln(x) dx$%
\item%
D. $u=x^2 \ln(x)$, $dv=dx$%
\end{itemize}%
\vspace{0.5cm}%
\item%
After the first application of integration by parts to $\int x^2 \sin(x) dx$ using $u=x^2$ and $dv=\sin(x)dx$, the resulting expression is: (3 points)%
\begin{itemize}%
\item%
A. $-x^2 \cos(x) + 2 \int x \cos(x) dx$%
\item%
B. $x^2 \cos(x) - 2 \int x \cos(x) dx$%
\item%
C. $-x^2 \cos(x) + \int 2x \sin(x) dx$%
\item%
D. $2x \sin(x) - \int x^2 \cos(x) dx$%
\end{itemize}%
\vspace{0.5cm}%
\item%
Using the reduction formula $I_n = \int \cos^n(x) dx = \frac{1}{n}\cos^{n-1}(x)\sin(x) + \frac{n-1}{n}I_{n-2}$, calculate $I_3 = \int \cos^3(x) dx$. (3 points)%
\begin{itemize}%
\item%
A. $\frac{1}{3}\cos^2(x)\sin(x) + \frac{2}{3}\sin(x) + C$%
\item%
B. $\frac{1}{3}\cos^2(x)\sin(x) + \frac{2}{3}\cos(x) + C$%
\item%
C. $\cos^3(x)\sin(x) + 3\int \cos^2(x) dx$%
\item%
D. $\sin(x) - \frac{1}{3}\sin^3(x) + C$%
\end{itemize}%
\vspace{0.5cm}%
\item%
Find the general solution to the separable differential equation $\frac{dy}{dx} = 4x^3 y$. (3 points)%
\begin{itemize}%
\item%
A. $y = x^4 + C$%
\item%
B. $y = C e^{x^4}$%
\item%
C. $y = e^{4x^3} + C$%
\item%
D. $\ln|y| = 4x^3 + C$%
\end{itemize}%
\vspace{0.5cm}%
\item%
Solve the initial value problem $\frac{dy}{dx} = \frac{x}{y}$ with the condition $y(0)=2$. (3 points)%
\begin{itemize}%
\item%
A. $y = x+2$%
\item%
B. $y = \sqrt{x^2+4}$%
\item%
C. $y = \sqrt{x^2+2}$%
\item%
D. $y^2 = x^2 + 4$%
\end{itemize}%
\vspace{0.5cm}%
\item%
The general solution to the separable equation $x(y^2-1) + y(x^2-1)\frac{dy}{dx} = 0$ is: (3 points)%
\begin{itemize}%
\item%
A. $\ln|y^2-1| + \ln|x^2-1| = C$%
\item%
B. $(1-y^2)(1-x^2) = C$%
\item%
C. $y^2 x^2 = C$%
\item%
D. $\frac{y^2-1}{x^2-1} = C$%
\end{itemize}%
\vspace{0.5cm}%
\item%
According to Kirchhoff's laws, the differential equation describing the charge $Q(t)$ in a series RLC circuit with voltage source $V(t)$ is: (3 points)%
\begin{itemize}%
\item%
A. $R\frac{dQ}{dt} + L\frac{d^2Q}{dt^2} + C Q = V(t)$%
\item%
B. $L\frac{d^2Q}{dt^2} + R\frac{dQ}{dt} + \frac{1}{C}Q = V(t)$%
\item%
C. $L\frac{dQ}{dt} + R Q + \frac{Q}{C} = V(t)$%
\item%
D. $R\frac{d^2Q}{dt^2} + L\frac{dQ}{dt} + \frac{1}{C}Q = V(t)$%
\end{itemize}%
\vspace{0.5cm}%
\item%
What is the order of the differential equation describing the motion of a mass $m$ subject to Newton's Second Law ($F=ma$), where $a$ is acceleration? (3 points)%
\begin{itemize}%
\item%
A. Zero order%
\item%
B. First order%
\item%
C. Second order%
\item%
D. Third order%
\end{itemize}%
\vspace{0.5cm}%
\item%
A radioactive substance decays at a rate proportional to the amount $N(t)$ remaining. If $k$ is the decay constant, formulate the first-order ODE describing this process. (3 points)%
\begin{itemize}%
\item%
A. $\frac{dN}{dt} = k t$%
\item%
B. $\frac{dN}{dt} = k N$%
\item%
C. $\frac{dN}{dt} = -k N$%
\item%
D. $\frac{d^2N}{dt^2} = -k N$%
\end{itemize}%
\vspace{0.5cm}%
\item%
In the context of differential equations, the 'Order' of an ODE is defined as: (3 points)%
\begin{itemize}%
\item%
A. The highest power to which the dependent variable is raised.%
\item%
B. The number of independent variables involved.%
\item%
C. The order of the highest derivative that occurs in the equation.%
\item%
D. The degree of the polynomial formed by the derivatives.%
\end{itemize}%
\vspace{0.5cm}%
\item%
A first-order ODE $\frac{dy}{dx} = f(x, y)$ is classified as separable if the function $f(x, y)$ can be expressed in which mathematical form? (3 points)%
\begin{itemize}%
\item%
A. $f(x, y) = a(x) + b(y)$%
\item%
B. $f(x, y) = a(x)b(y)$%
\item%
C. $f(x, y) = a(x)y + b(x)$%
\item%
D. $f(x, y) = a(x)/y^2$%
\end{itemize}%
\vspace{0.5cm}%
\item%
Which integration technique is most appropriate and efficient for evaluating the integral $\int x^2 \sqrt{x^3 + 5} dx$? (3 points)%
\begin{itemize}%
\item%
A. Integration by Parts, $u=x^2$%
\item%
B. Trigonometric Substitution%
\item%
C. Substitution, $u=x^3 + 5$%
\item%
D. Partial Fractions%
\end{itemize}%
\vspace{0.5cm}%
\item%
Consider the integral $\int \frac{x}{x^2+1} dx$. Which technique is required to solve this integral? (3 points)%
\begin{itemize}%
\item%
A. Integration by Parts%
\item%
B. Trigonometric Substitution%
\item%
C. Simple $u$-Substitution%
\item%
D. Reduction Formula%
\end{itemize}%
\vspace{0.5cm}%
\item%
When applying integration by parts to $\int (2x-1) \ln(x^2+1) dx$, the preferred choice is $u = \ln(x^2+1)$ and $dv = (2x-1) dx$. What is the primary mathematical reason for choosing $u$ as the logarithmic function? (3 points)%
\begin{itemize}%
\item%
A. Logarithmic functions are always easier to integrate than polynomials.%
\item%
B. The derivative of $u=\ln(x^2+1)$ is a rational function $\frac{2x}{x^2+1}$, which simplifies the subsequent integral $\int v du$.%
\item%
C. The integral of $dv=(2x-1)dx$ is complex, so it must be chosen as $dv$.%
\item%
D. This choice ensures that $v$ is a constant, simplifying the $uv$ term.%
\end{itemize}%
\vspace{0.5cm}%
\item%
Evaluate the indefinite integral using Integration by Parts (IBP). Show all steps clearly:
$$\int x^2 e^{3x} dx$$ (5 points)%
\vspace{4cm}%
\vspace{0.5cm}%
\item%
Find the general solution to the first-order separable differential equation:
$$\frac{dy}{dx} = \frac{x^2}{y^3}$$ (5 points)%
\vspace{4cm}%
\vspace{0.5cm}%
\item%
A mass $m$ is attached to a spring (constant $k$) and is subjected to a damping force proportional to its velocity $v(t)$, with damping coefficient $\gamma$. If the mass is also driven by an external force $F(t)$, formulate the second-order Ordinary Differential Equation (ODE) that describes the displacement $x(t)$ of the mass from its equilibrium position. (5 points)%
\vspace{4cm}%
\vspace{0.5cm}%
\item%
Define the mathematical concept of the **order** of an Ordinary Differential Equation (ODE). Based on this definition, state the order of the ODE describing the charge $Q(t)$ in a series RLC circuit:
$$L\frac{d^2Q}{dt^2} + R\frac{dQ}{dt} + \frac{1}{C}Q = V(t)$$ (5 points)%
\vspace{4cm}%
\vspace{0.5cm}%
\item%
Use Integration by Parts to evaluate the indefinite integral. Hint: You may need algebraic manipulation after the first step.
$$\int \ln(x^2+1) dx$$ (5 points)%
\vspace{4cm}%
\vspace{0.5cm}%
\item%
Evaluate the indefinite integral using the technique of Integration by Parts. Show all steps, including the selection of $u$ and $dv$ for each application.
$$\int x^2 \cos(4x) dx$$ (10 points)%
\vspace{8cm}%
\vspace{0.5cm}%
\item%
A mass $m$ is attached to a spring with spring constant $k$. The mass is also subjected to a damping force proportional to its velocity, with damping coefficient $c$. An external driving force $F(t) = F_0 \sin(\omega t)$ is applied to the system.

a) Formulate the second-order ordinary differential equation (ODE) that describes the displacement $x(t)$ of the mass from its equilibrium position, based on Newton's Second Law.

b) If $m=2$ kg, $c=4$ Ns/m, $k=50$ N/m, $F_0=10$ N, and $\omega=3$ rad/s, write down the specific ODE for this system.

c) State the order and linearity of the resulting ODE. (10 points)%
\vspace{8cm}%
\vspace{0.5cm}%
\item%
Find the general solution $y(x)$ for the following first-order separable differential equation. Ensure you show the separation of variables and the integration steps clearly.
$$\frac{dy}{dx} = \frac{e^{2x} \sqrt{y}}{y^2}$$ (10 points)%
\vspace{8cm}%
\vspace{0.5cm}%
\end{enumerate}

%
\newpage%
\section*{Solution Key & Grading Rubric}%
\label{sec:SolutionKeyGradingRubric}%
\textbf{Confidential - Instructor Use Only}%
\vspace{0.5cm}%
\begin{enumerate}%
\item%
\textbf{M-1}%
 \ \ \textbf{Model Answer:} A. $\frac{x}{3}e^{3x} - \frac{1}{9}e^{3x} + C$%
 \\ \textit{Grading Notes:} 1 pt for correct selection of $u=x$ and $dv=e^{3x}dx$. 1 pt for correct application of the IBP formula: $\frac{x}{3}e^{3x} - \int \frac{1}{3}e^{3x} dx$. 1 pt for accurate final integration and algebraic simplification.%
\vspace{0.3cm}%
\item%
\textbf{M-2}%
 \ \ \textbf{Model Answer:} B. $u=\ln(x)$, $dv=x^2 dx$%
 \\ \textit{Grading Notes:} 3 pts for identifying the correct heuristic (LATE/LIATE rule) that dictates choosing $u=\ln(x)$ because its derivative is simpler, while $dv=x^2 dx$ is easily integrable. 0 pts if $u$ is chosen as $x^2$, leading to a more complex integral.%
\vspace{0.3cm}%
\item%
\textbf{M-3}%
 \ \ \textbf{Model Answer:} A. $-x^2 \cos(x) + 2 \int x \cos(x) dx$%
 \\ \textit{Grading Notes:} 1 pt for correctly finding $du=2x dx$ and $v=-\cos(x)$. 2 pts for correctly substituting into the formula $\int u dv = uv - \int v du$, ensuring correct signs: $(-x^2 \cos(x)) - \int (-\cos(x))(2x dx) = -x^2 \cos(x) + 2 \int x \cos(x) dx$.%
\vspace{0.3cm}%
\item%
\textbf{M-4}%
 \ \ \textbf{Model Answer:} A. $\frac{1}{3}\cos^2(x)\sin(x) + \frac{2}{3}\sin(x) + C$%
 \\ \textit{Grading Notes:} 1 pt for correctly applying the formula for $n=3$: $I_3 = \frac{1}{3}\cos^2(x)\sin(x) + \frac{2}{3}I_1$. 1 pt for correctly identifying $I_1 = \int \cos(x) dx = \sin(x)$. 1 pt for the correct final expression.%
\vspace{0.3cm}%
\item%
\textbf{M-5}%
 \ \ \textbf{Model Answer:} B. $y = C e^{x^4}$%
 \\ \textit{Grading Notes:} 1 pt for correctly separating variables: $\int \frac{1}{y} dy = \int 4x^3 dx$. 1 pt for correct integration: $\ln|y| = x^4 + C_1$. 1 pt for solving for $y$ and correctly handling the constant of integration ($y = e^{x^4+C_1} = e^{C_1}e^{x^4} = C e^{x^4}$).%
\vspace{0.3cm}%
\item%
\textbf{M-6}%
 \ \ \textbf{Model Answer:} B. $y = \sqrt{x^2+4}$%
 \\ \textit{Grading Notes:} 1 pt for finding the general solution implicitly: $y^2 = x^2 + K$. 1 pt for using the initial condition $y(0)=2$ to find the constant $K=4$. 1 pt for solving explicitly for $y$, choosing the positive root based on the initial condition.%
\vspace{0.3cm}%
\item%
\textbf{M-7}%
 \ \ \textbf{Model Answer:} B. $(1-y^2)(1-x^2) = C$%
 \\ \textit{Grading Notes:} 1 pt for correct separation: $\frac{y}{y^2-1} dy = -\frac{x}{x^2-1} dx$. 1 pt for correct integration: $\frac{1}{2}\ln|y^2-1| = -\frac{1}{2}\ln|x^2-1| + C'$. 1 pt for correct algebraic manipulation to obtain the final implicit form, recognizing that $|(y^2-1)(x^2-1)| = K$ is equivalent to $(1-y^2)(1-x^2) = C$ (by absorbing signs and constants).%
\vspace{0.3cm}%
\item%
\textbf{M-8}%
 \ \ \textbf{Model Answer:} B. $L\frac{d^2Q}{dt^2} + R\frac{dQ}{dt} + \frac{1}{C}Q = V(t)$%
 \\ \textit{Grading Notes:} 3 pts for correctly translating the physical components (Inductor voltage $L\frac{dI}{dt} = L\frac{d^2Q}{dt^2}$, Resistor voltage $RI = R\frac{dQ}{dt}$, Capacitor voltage $\frac{Q}{C}$) into the correct second-order ODE structure.%
\vspace{0.3cm}%
\item%
\textbf{M-9}%
 \ \ \textbf{Model Answer:} C. Second order%
 \\ \textit{Grading Notes:} 3 pts for recognizing that acceleration ($a$) is the second derivative of displacement ($x$) with respect to time ($a = \frac{d^2x}{dt^2}$), making the resulting ODE second order.%
\vspace{0.3cm}%
\item%
\textbf{M-10}%
 \ \ \textbf{Model Answer:} C. $\frac{dN}{dt} = -k N$%
 \\ \textit{Grading Notes:} 1 pt for recognizing that the rate is proportional to $N$. 2 pts for correctly identifying that decay implies a negative rate of change, requiring the negative sign: $\frac{dN}{dt} = -k N$.%
\vspace{0.3cm}%
\item%
\textbf{M-11}%
 \ \ \textbf{Model Answer:} C. The order of the highest derivative that occurs in the equation.%
 \\ \textit{Grading Notes:} 3 pts for accurate recall and definition of the order of an ODE.%
\vspace{0.3cm}%
\item%
\textbf{M-12}%
 \ \ \textbf{Model Answer:} B. $f(x, y) = a(x)b(y)$%
 \\ \textit{Grading Notes:} 3 pts for accurate recall of the condition for separability, allowing the equation to be rewritten as $\frac{1}{b(y)} dy = a(x) dx$.%
\vspace{0.3cm}%
\item%
\textbf{M-13}%
 \ \ \textbf{Model Answer:} C. Substitution, $u=x^3 + 5$%
 \\ \textit{Grading Notes:} 3 pts for analyzing the structure and recognizing the presence of $x^2$ as a factor of the derivative of the inner function $x^3+5$. This indicates a simple $u$-substitution is the most efficient method.%
\vspace{0.3cm}%
\item%
\textbf{M-14}%
 \ \ \textbf{Model Answer:} C. Simple $u$-Substitution%
 \\ \textit{Grading Notes:} 3 pts for recognizing that $u=x^2+1$ yields $du=2x dx$, making the integral solvable by simple substitution ($\frac{1}{2}\int \frac{1}{u} du$).%
\vspace{0.3cm}%
\item%
\textbf{M-15}%
 \ \ \textbf{Model Answer:} B. The derivative of $u=\ln(x^2+1)$ is a rational function $\frac{2x}{x^2+1}$, which simplifies the subsequent integral $\int v du$.%
 \\ \textit{Grading Notes:} 3 pts for demonstrating analytical understanding of the IBP heuristic: choosing $u$ such that $du$ is simpler or easier to handle when multiplied by $v$. The derivative of the logarithm is algebraic and often leads to a solvable integral.%
\vspace{0.3cm}%
\item%
\textbf{S-1}%
 \ \ \textbf{Model Answer:} We use Integration by Parts twice.
Step 1: Let $u_1 = x^2$, $dv_1 = e^{3x} dx$. Then $du_1 = 2x dx$, $v_1 = \frac{1}{3}e^{3x}$.
$$\int x^2 e^{3x} dx = \frac{1}{3}x^2 e^{3x} - \int \frac{1}{3}e^{3x} (2x) dx = \frac{1}{3}x^2 e^{3x} - \frac{2}{3} \int x e^{3x} dx$$
Step 2: Evaluate $\int x e^{3x} dx$. Let $u_2 = x$, $dv_2 = e^{3x} dx$. Then $du_2 = dx$, $v_2 = \frac{1}{3}e^{3x}$.
$$\int x e^{3x} dx = \frac{1}{3}x e^{3x} - \int \frac{1}{3}e^{3x} dx = \frac{1}{3}x e^{3x} - \frac{1}{9}e^{3x}$$
Substituting back:
$$\int x^2 e^{3x} dx = \frac{1}{3}x^2 e^{3x} - \frac{2}{3} \left( \frac{1}{3}x e^{3x} - \frac{1}{9}e^{3x} \right) + C$$
$$\int x^2 e^{3x} dx = \frac{1}{3}x^2 e^{3x} - \frac{2}{9}x e^{3x} + \frac{2}{27}e^{3x} + C$$%
 \\ \textit{Grading Notes:} 1 pt for correct first IBP setup ($u_1, dv_1$). 2 pts for correctly executing the first IBP and setting up the second integral. 1 pt for correctly executing the second IBP. 1 pt for the final simplified algebraic expression and the constant of integration $C$.%
\vspace{0.3cm}%
\item%
\textbf{S-2}%
 \ \ \textbf{Model Answer:} Separate the variables:
$$y^3 dy = x^2 dx$$
Integrate both sides:
$$\int y^3 dy = \int x^2 dx$$
$$\frac{y^4}{4} = \frac{x^3}{3} + C_1$$
To clear fractions and combine the constant:
$$3y^4 = 4x^3 + 12C_1$$
Let $C = 12C_1$. The general solution is:
$$3y^4 - 4x^3 = C$$%
 \\ \textit{Grading Notes:} 1 pt for correctly separating the variables. 2 pts for correctly integrating both sides (LHS: $y^4/4$, RHS: $x^3/3$). 1 pt for including the constant of integration. 1 pt for the final, algebraically correct general solution form.%
\vspace{0.3cm}%
\item%
\textbf{S-3}%
 \ \ \textbf{Model Answer:} According to Newton's Second Law, $F_{net} = ma$, where $a = \frac{d^2x}{dt^2}$.
The forces acting on the mass are:
1. Spring Force (Restoring): $F_s = -kx$
2. Damping Force: $F_d = -\gamma v = -\gamma \frac{dx}{dt}$
3. External Force: $F_e = F(t)$
Summing the forces:
$$m \frac{d^2x}{dt^2} = -kx - \gamma \frac{dx}{dt} + F(t)$$
Rearranging into standard form for a damped, driven harmonic oscillator:
$$m \frac{d^2x}{dt^2} + \gamma \frac{dx}{dt} + kx = F(t)$$%
 \\ \textit{Grading Notes:} 1 pt for stating Newton's Second Law ($F_{net} = ma$ or $m \frac{d^2x}{dt^2}$). 1 pt for correctly identifying the spring force term ($kx$). 1 pt for correctly identifying the damping force term ($\gamma \frac{dx}{dt}$). 2 pts for assembling the final, correct second-order linear ODE in standard form.%
\vspace{0.3cm}%
\item%
\textbf{S-4}%
 \ \ \textbf{Model Answer:} The **order** of an Ordinary Differential Equation is defined as the order of the highest derivative that appears in the equation.
In the given RLC circuit equation, the highest derivative present is the second derivative, $\frac{d^2Q}{dt^2}$.
Therefore, the order of this ODE is **2 (Second Order)**.%
 \\ \textit{Grading Notes:} 2 pts for the correct definition of the order of an ODE (the order of the highest derivative). 1 pt for identifying the highest derivative in the RLC equation. 2 pts for correctly stating the order is 2.%
\vspace{0.3cm}%
\item%
\textbf{S-5}%
 \ \ \textbf{Model Answer:} We use Integration by Parts: $\int u dv = uv - \int v du$.
Let $u = \ln(x^2+1)$ and $dv = dx$.
Then $du = \frac{2x}{x^2+1} dx$ and $v = x$.
$$\int \ln(x^2+1) dx = x \ln(x^2+1) - \int x \left( \frac{2x}{x^2+1} \right) dx$$
$$= x \ln(x^2+1) - 2 \int \frac{x^2}{x^2+1} dx$$
We rewrite the integrand using algebraic division/manipulation:
$$\frac{x^2}{x^2+1} = \frac{(x^2+1) - 1}{x^2+1} = 1 - \frac{1}{x^2+1}$$
Substituting back and integrating:
$$= x \ln(x^2+1) - 2 \int \left( 1 - \frac{1}{x^2+1} \right) dx$$
$$= x \ln(x^2+1) - 2 \left( x - \arctan(x) \right) + C$$
$$= x \ln(x^2+1) - 2x + 2\arctan(x) + C$$%
 \\ \textit{Grading Notes:} 1 pt for the correct IBP setup ($u=\ln(x^2+1)$, $dv=dx$). 2 pts for correctly executing the IBP leading to the integral of $\frac{x^2}{x^2+1}$. 1 pt for the algebraic manipulation to integrate $\frac{x^2}{x^2+1}$. 1 pt for the final correct integration, including $\arctan(x)$ and the constant $C$.%
\vspace{0.3cm}%
\item%
\textbf{P-1}%
 \ \ \textbf{Model Answer:} This integral requires two applications of Integration by Parts (IBP), $\int u dv = uv - \int v du$.

**First Application:**
Let $u_1 = x^2$ and $dv_1 = \cos(4x) dx$.
Then $du_1 = 2x dx$ and $v_1 = \frac{1}{4} \sin(4x)$.

$$\int x^2 \cos(4x) dx = x^2 \left(\frac{1}{4} \sin(4x)\right) - \int \left(\frac{1}{4} \sin(4x)\right) (2x dx)$$
$$= \frac{1}{4} x^2 \sin(4x) - \frac{1}{2} \int x \sin(4x) dx$$

**Second Application (on $\int x \sin(4x) dx$):**
Let $u_2 = x$ and $dv_2 = \sin(4x) dx$.
Then $du_2 = dx$ and $v_2 = -\frac{1}{4} \cos(4x)$.

$$\int x \sin(4x) dx = x \left(-\frac{1}{4} \cos(4x)\right) - \int \left(-\frac{1}{4} \cos(4x)\right) dx$$
$$= -\frac{1}{4} x \cos(4x) + \frac{1}{4} \int \cos(4x) dx$$
$$= -\frac{1}{4} x \cos(4x) + \frac{1}{4} \left(\frac{1}{4} \sin(4x)\right) + C'$$
$$= -\frac{1}{4} x \cos(4x) + \frac{1}{16} \sin(4x) + C'$$

**Final Solution:**
Substitute the result of the second integral back into the first equation:
$$\int x^2 \cos(4x) dx = \frac{1}{4} x^2 \sin(4x) - \frac{1}{2} \left[ -\frac{1}{4} x \cos(4x) + \frac{1}{16} \sin(4x) \right] + C$$
$$\int x^2 \cos(4x) dx = \frac{1}{4} x^2 \sin(4x) + \frac{1}{8} x \cos(4x) - \frac{1}{32} \sin(4x) + C$$%
 \\ \textit{Grading Notes:} Total 10 points.
1. (2 pts) Correct selection of $u_1$ and $dv_1$ for the first IBP ($u_1=x^2$).
2. (3 pts) Successful execution of the first IBP, leading to the second integral $\int x \sin(4x) dx$.
3. (2 pts) Correct selection of $u_2$ and $dv_2$ for the second IBP ($u_2=x$).
4. (2 pts) Successful execution and integration of the second IBP.
5. (1 pt) Correctly combining all terms and including the constant of integration $+C$. Algebraic precision is required for full credit on steps 2, 4, and 5.%
\vspace{0.3cm}%
\item%
\textbf{P-2}%
 \ \ \textbf{Model Answer:} a) According to Newton's Second Law, $F_{net} = ma$. The forces acting on the mass are:
1. Spring Force (Restoring): $F_s = -kx$
2. Damping Force: $F_d = -c \frac{dx}{dt}$
3. External Force (Driving): $F_e = F_0 \sin(\omega t)$

Since $a = \frac{d^2x}{dt^2}$, the ODE is:
$$m \frac{d^2x}{dt^2} = -kx - c \frac{dx}{dt} + F_0 \sin(\omega t)$$
Rearranging into standard form:
$$m \frac{d^2x}{dt^2} + c \frac{dx}{dt} + kx = F_0 \sin(\omega t)$$

b) Substituting the given values ($m=2, c=4, k=50, F_0=10, \omega=3$):
$$2 \frac{d^2x}{dt^2} + 4 \frac{dx}{dt} + 50x = 10 \sin(3t)$$

c) Order: The highest derivative is the second derivative, so the ODE is **Second Order**.
Linearity: The dependent variable $x$ and its derivatives appear only to the first power, and the coefficients ($2, 4, 50$) depend only on constants (or the independent variable $t$, which is not the case here). Thus, the ODE is **Linear**.%
 \\ \textit{Grading Notes:} Total 10 points.
1. (3 pts) Correctly identifying and summing the forces based on Newton's Second Law, leading to the general form of the ODE (Part a).
2. (3 pts) Correctly substituting all numerical values into the general ODE (Part b).
3. (2 pts) Correctly stating the order of the ODE (Second Order).
4. (2 pts) Correctly stating the linearity of the ODE (Linear).%
\vspace{0.3cm}%
\item%
\textbf{P-3}%
 \ \ \textbf{Model Answer:} The differential equation is $\frac{dy}{dx} = \frac{e^{2x} \sqrt{y}}{y^2}$.

**Step 1: Separation of Variables**
We rearrange the equation to group $y$ terms with $dy$ and $x$ terms with $dx$:
$$\frac{y^2}{\sqrt{y}} dy = e^{2x} dx$$
Simplifying the $y$ term using exponent rules ($y^2 / y^{1/2} = y^{2 - 1/2} = y^{3/2}$):
$$y^{3/2} dy = e^{2x} dx$$

**Step 2: Integration**
Integrate both sides:
$$\int y^{3/2} dy = \int e^{2x} dx$$

Integrating the left side:
$$\int y^{3/2} dy = \frac{y^{3/2 + 1}}{3/2 + 1} = \frac{y^{5/2}}{5/2} = \frac{2}{5} y^{5/2} + C_1$$

Integrating the right side (using substitution $u=2x$):
$$\int e^{2x} dx = \frac{1}{2} e^{2x} + C_2$$

**Step 3: General Solution**
Equating the results and combining constants $C = C_2 - C_1$:
$$\frac{2}{5} y^{5/2} = \frac{1}{2} e^{2x} + C$$

**Step 4: Explicit Solution (Optional, but preferred for clarity)**
$$y^{5/2} = \frac{5}{2} \left( \frac{1}{2} e^{2x} + C \right)$$
$$y^{5/2} = \frac{5}{4} e^{2x} + C_3 \quad \text{where } C_3 = \frac{5}{2} C$$
$$y(x) = \left( \frac{5}{4} e^{2x} + C_3 \right)^{2/5}$$%
 \\ \textit{Grading Notes:} Total 10 points.
1. (3 pts) Correct algebraic manipulation and separation of variables, resulting in $y^{3/2} dy = e^{2x} dx$.
2. (3 pts) Correct integration of the left side (y-term), yielding $\frac{2}{5} y^{5/2}$.
3. (3 pts) Correct integration of the right side (x-term), yielding $\frac{1}{2} e^{2x}$.
4. (1 pt) Correctly combining the integrated terms and including the single constant of integration $C$ (or $C_3$).%
\vspace{0.3cm}%
\end{enumerate}

%
\end{document}