\documentclass[11pt]{article}%
\usepackage[T1]{fontenc}%
\usepackage[utf8]{inputenc}%
\usepackage{lmodern}%
\usepackage{textcomp}%
\usepackage{lastpage}%
\usepackage[margin=1in]{geometry}%
\usepackage{amsmath}%
\usepackage{amssymb}%
\usepackage{fancyhdr}%
\usepackage{titlesec}%
%
%
%
\begin{document}%
\normalsize%
\begin{titlepage}%
\centering%
\vspace*{1cm}%
{\Huge \textbf{UNIVERSITY OF TORONTO} \par}%
\vspace{1.5cm}%
{\Large \textbf{Final Exam: Science Fiction and Fantasy Writing Lecture - Introductory Material} \par}%
\vspace{0.5cm}%
{\large \today \par}%
\vspace{2cm}%
\textbf{INSTRUCTIONS TO CANDIDATES} \par%
\vspace{0.5cm}%
\textit{(Professional and direct. Instructions will emphasize critical thinking and the precise application of the course's philosophical framework, treating the student as an aspiring professional writer. Tone) Conceptual Short Answer and Analytical Essay. The assessment will test understanding of foundational writing philosophies, process definitions, and the professional mindset required for a career in SF/F. Exam. Please answer all questions.}%
\vspace{3cm}%
\textbf{DO NOT OPEN THIS BOOKLET UNTIL TOLD TO DO SO} \par%
\vfill%
{\large Total Points: 100 \par}%
\end{titlepage}%
\newpage%
\section*{Questions}%
\label{sec:Questions}%
\begin{enumerate}%
\item%
According to the lecture, which term best describes a writer who prefers extensive pre-planning, detailed world-building notes, and a structured outline before beginning the first draft? (3 points)%
\begin{itemize}%
\item%
Gardener%
\item%
Discovery Writer%
\item%
Architect%
\item%
Revisionist%
\end{itemize}%
\vspace{0.5cm}%
\item%
A 'Discovery Writer' (or 'Gardener') primarily focuses on which aspect of the writing process? (3 points)%
\begin{itemize}%
\item%
A. Completing a detailed 50{-}page outline before starting.%
\item%
B. Nurturing the story organically, allowing characters and plot to emerge during drafting.%
\item%
C. Rigorously revising the first three chapters until they are perfect.%
\item%
D. Focusing solely on market trends and commercial viability.%
\end{itemize}%
\vspace{0.5cm}%
\item%
The instructor notes that most successful writers fall into which category regarding the Discovery Writer/Outline Writer spectrum? (3 points)%
\begin{itemize}%
\item%
A. Pure Discovery Writers, relying entirely on instinct.%
\item%
B. Pure Outline Writers, adhering strictly to detailed plans.%
\item%
C. A blend of both, using outlines for structure but allowing for discovery during the draft.%
\item%
D. Writers who only outline after the first draft is complete.%
\end{itemize}%
\vspace{0.5cm}%
\item%
The instructor uses the metaphor of a 'Chef' versus a 'Cook' to illustrate the desired learning outcome of the course. What does becoming a 'Chef' in writing signify? (3 points)%
\begin{itemize}%
\item%
A. Strictly following the established rules (recipes) of successful authors.%
\item%
B. Understanding the underlying principles and tools of writing to create specific, desired effects.%
\item%
C. Prioritizing speed and output over quality.%
\item%
D. Only writing genres that are currently selling well in the market.%
\end{itemize}%
\vspace{0.5cm}%
\item%
In the context of the 'Chef vs. Cook' metaphor, a 'Cook' is analogous to a writer who: (3 points)%
\begin{itemize}%
\item%
A. Masters the art of revision and editing.%
\item%
B. Only follows specific writing advice (recipes) without understanding why they work.%
\item%
C. Writes primarily for personal fulfillment, ignoring commercial success.%
\item%
D. Collaborates extensively in writing groups.%
\end{itemize}%
\vspace{0.5cm}%
\item%
'Survivorship Bias,' as discussed in the lecture, primarily refers to the error in judgment where one: (3 points)%
\begin{itemize}%
\item%
A. Over{-}revises early chapters, preventing forward progress.%
\item%
B. Focuses too much on the business side of writing rather than the artistic side.%
\item%
C. Overvalues the advice of successful individuals while ignoring the larger population of failures or the role of luck.%
\item%
D. Writes only what they enjoy, regardless of market demand.%
\end{itemize}%
\vspace{0.5cm}%
\item%
The primary implication of recognizing 'Survivorship Bias' for an aspiring writer is that they should: (3 points)%
\begin{itemize}%
\item%
A. Assume all successful advice is universally applicable.%
\item%
B. Disregard all advice from published authors.%
\item%
C. Understand that success stories often omit the randomness and failure inherent in the process, requiring critical evaluation of advice.%
\item%
D. Focus exclusively on statistical data regarding sales figures.%
\end{itemize}%
\vspace{0.5cm}%
\item%
According to the instructor's advice for new writers, what should be the primary focus during the initial years of development? (3 points)%
\begin{itemize}%
\item%
A. Securing an agent and maximizing sales of the first novel.%
\item%
B. Learning the process, practicing consistently, and completing multiple books to find their voice.%
\item%
C. Only writing stories that fit current market trends (e.g., Grimdark).%
\item%
D. Spending all effort on networking with top editors.%
\end{itemize}%
\vspace{0.5cm}%
\item%
Why does the instructor emphasize that revision is a skill as difficult and important to learn as writing the first draft? (3 points)%
\begin{itemize}%
\item%
A. Revision is necessary only if the writer is a Discovery Writer.%
\item%
B. Revision is the primary way to learn market trends.%
\item%
C. Revision requires a different set of critical and analytical skills than drafting, and poor revision skills can delay publication significantly (as seen in the instructor's own experience).%
\item%
D. Revision is necessary only to satisfy the demands of a writing group.%
\end{itemize}%
\vspace{0.5cm}%
\item%
The instructor advises that a writer should separate their personality into two roles: the Artist and the Businessperson. When should the Artist persona be primarily in control? (3 points)%
\begin{itemize}%
\item%
A. During the contract negotiation phase.%
\item%
B. While writing the first draft.%
\item%
C. When calculating royalties and taxes.%
\item%
D. When reading reviews and critiques.%
\end{itemize}%
\vspace{0.5cm}%
\item%
The instructor compares a new writer's early works (the first few books) to what musical concept, emphasizing practice and development over immediate performance success? (3 points)%
\begin{itemize}%
\item%
A. Writing a symphony.%
\item%
B. Learning scales on a piano.%
\item%
C. Performing a concert solo.%
\item%
D. Conducting an orchestra.%
\end{itemize}%
\vspace{0.5cm}%
\item%
What specific danger related to revision does the instructor warn new writers about? (3 points)%
\begin{itemize}%
\item%
A. Not revising enough before submitting.%
\item%
B. Over{-}revising the first few chapters repeatedly, thereby preventing the completion of the book.%
\item%
C. Only revising based on the advice of a single editor.%
\item%
D. Failing to use strict \$\textbackslash{}LaTeX\$ for mathematical concepts.%
\end{itemize}%
\vspace{0.5cm}%
\item%
From a professional standpoint, what is the most critical factor that distinguishes a professional writer from a non-professional writer, according to the lecture's context? (3 points)%
\begin{itemize}%
\item%
A. Having a degree in English Literature.%
\item%
B. The commitment to treating writing as a career and consistently producing work, regardless of immediate sales.%
\item%
C. Writing exclusively in high{-}fantasy or science fiction genres.%
\item%
D. The ability to write 10,000 words per day.%
\end{itemize}%
\vspace{0.5cm}%
\item%
Regarding story concepts and premises, the instructor asserts that 'Ideas are cheap.' What does this statement imply about the writing process? (3 points)%
\begin{itemize}%
\item%
A. Writers should constantly seek new, unique ideas.%
\item%
B. The true value lies in the execution, craft, and revision of the story, not the initial concept.%
\item%
C. Ideas must be protected via strict copyright immediately.%
\item%
D. Only commercially successful ideas are worthwhile.%
\end{itemize}%
\vspace{0.5cm}%
\item%
The instructor gives specific advice to Discovery Writers participating in writing groups. What is the key caution for them? (3 points)%
\begin{itemize}%
\item%
A. They should only join groups composed entirely of other Discovery Writers.%
\item%
B. They must finish the entire draft before taking extensive critique, as early feedback might derail the organic discovery process.%
\item%
C. They should defend their creative choices vigorously against all criticism.%
\item%
D. They should only accept prescriptive feedback, not descriptive feedback.%
\end{itemize}%
\vspace{0.5cm}%
\item%
Differentiate between the "Gardener" and "Architect" writing philosophies as described in the lecture, and explain how the instructor suggests most successful writers relate to this dichotomy. (5 points)%
\vspace{4cm}%
\vspace{0.5cm}%
\item%
Explain the "Chef vs. Cook" metaphor used in the lecture. Specifically, what is the desired learning outcome of the course based on this metaphor, and why is being a "Chef" considered superior for a professional writer? (5 points)%
\vspace{4cm}%
\vspace{0.5cm}%
\item%
Define 'Survivorship Bias' in the context of writing advice. Analyze one specific implication this bias has for a new writer interpreting the success stories of established authors. (5 points)%
\vspace{4cm}%
\vspace{0.5cm}%
\item%
Synthesize the instructor's advice regarding the professional approach to early writing careers. Specifically, explain why a new writer should prioritize 'process and practice' (like learning scales) over immediate sales or excessive revision of their first few manuscripts. (5 points)%
\vspace{4cm}%
\vspace{0.5cm}%
\item%
Define and differentiate the 'Discovery Writer' and the 'Outline Writer.' Provide one specific challenge or drawback associated with each approach during the drafting phase. (5 points)%
\vspace{4cm}%
\vspace{0.5cm}%
\item%
Analyze the relationship between the 'Gardener/Architect' dichotomy and the 'Chef/Cook' metaphor introduced in the lecture. Define both pairs of terms using course terminology. Explain why a writer who identifies primarily as a 'Gardener' must still strive to achieve the status of a 'Chef' to succeed professionally, according to the course philosophy. (10 points)%
\vspace{8cm}%
\vspace{0.5cm}%
\item%
The instructor warns students about 'Survivorship Bias' when evaluating writing advice and professional success. Define Survivorship Bias using course terminology. Then, explain how the instructor's core advice—to focus on the process of writing multiple books (e.g., the first six) rather than immediate sales or external validation—serves as a practical countermeasure against the psychological pitfalls associated with this bias. (10 points)%
\vspace{8cm}%
\vspace{0.5cm}%
\item%
The professional writing mindset requires specific approaches to revision and manuscript completion. Discuss two specific pieces of advice given in the lecture regarding this professional approach. Specifically, address (1) why new writers should avoid the temptation of over-revising their first chapters, and (2) how the concept of separating the 'Artist' persona from the 'Business' persona applies specifically to the revision and submission phase of a manuscript. (10 points)%
\vspace{8cm}%
\vspace{0.5cm}%
\end{enumerate}

%
\newpage%
\section*{Solution Key \& Grading Rubric}%
\label{sec:SolutionKeyGradingRubric}%
\textbf{Confidential - Instructor Use Only}%
\vspace{0.5cm}%
\begin{enumerate}%
\item%
\textbf{M-1}%
 \ \ \textbf{Model Answer:} Architect%
 \\ \textit{Grading Notes:} 3 points for correctly identifying the 'Architect' metaphor, which aligns with the Outline Writer philosophy.%
\vspace{0.3cm}%
\item%
\textbf{M-2}%
 \ \ \textbf{Model Answer:} B. Nurturing the story organically, allowing characters and plot to emerge during drafting.%
 \\ \textit{Grading Notes:} 3 points for defining the core characteristic of a Discovery Writer (writing to discover the story).%
\vspace{0.3cm}%
\item%
\textbf{M-3}%
 \ \ \textbf{Model Answer:} C. A blend of both, using outlines for structure but allowing for discovery during the draft.%
 \\ \textit{Grading Notes:} 3 points for recognizing the nuance that most writers utilize a mixed approach between the two extremes.%
\vspace{0.3cm}%
\item%
\textbf{M-4}%
 \ \ \textbf{Model Answer:} B. Understanding the underlying principles and tools of writing to create specific, desired effects.%
 \\ \textit{Grading Notes:} 3 points for applying the 'Chef' metaphor correctly, emphasizing mastery of principles (tools) over mere adherence to rules (recipes).%
\vspace{0.3cm}%
\item%
\textbf{M-5}%
 \ \ \textbf{Model Answer:} B. Only follows specific writing advice (recipes) without understanding why they work.%
 \\ \textit{Grading Notes:} 3 points for correctly identifying the 'Cook' as someone who relies on prescriptive advice without grasping the foundational mechanics.%
\vspace{0.3cm}%
\item%
\textbf{M-6}%
 \ \ \textbf{Model Answer:} C. Overvalues the advice of successful individuals while ignoring the larger population of failures or the role of luck.%
 \\ \textit{Grading Notes:} 3 points for accurately defining Survivorship Bias in the context of writing success and advice.%
\vspace{0.3cm}%
\item%
\textbf{M-7}%
 \ \ \textbf{Model Answer:} C. Understand that success stories often omit the randomness and failure inherent in the process, requiring critical evaluation of advice.%
 \\ \textit{Grading Notes:} 3 points for analyzing the practical consequence of Survivorship Bias—the need for critical evaluation of success narratives.%
\vspace{0.3cm}%
\item%
\textbf{M-8}%
 \ \ \textbf{Model Answer:} B. Learning the process, practicing consistently, and completing multiple books to find their voice.%
 \\ \textit{Grading Notes:} 3 points for synthesizing the advice regarding focusing on process and volume (writing multiple books) over immediate sales or external validation.%
\vspace{0.3cm}%
\item%
\textbf{M-9}%
 \ \ \textbf{Model Answer:} C. Revision requires a different set of critical and analytical skills than drafting, and poor revision skills can delay publication significantly (as seen in the instructor's own experience).%
 \\ \textit{Grading Notes:} 3 points for accurately reflecting the instructor's personal experience and emphasis on revision as a distinct, critical professional skill.%
\vspace{0.3cm}%
\item%
\textbf{M-10}%
 \ \ \textbf{Model Answer:} B. While writing the first draft.%
 \\ \textit{Grading Notes:} 3 points for correctly identifying that the creative (Artist) mindset should dominate the drafting phase, while the pragmatic (Businessperson) mindset handles the finished product.%
\vspace{0.3cm}%
\item%
\textbf{M-11}%
 \ \ \textbf{Model Answer:} B. Learning scales on a piano.%
 \\ \textit{Grading Notes:} 3 points for recalling the analogy used to stress the importance of foundational practice and volume in early career development.%
\vspace{0.3cm}%
\item%
\textbf{M-12}%
 \ \ \textbf{Model Answer:} B. Over-revising the first few chapters repeatedly, thereby preventing the completion of the book.%
 \\ \textit{Grading Notes:} 3 points for identifying the practical pitfall of 'polishing the first chapter' syndrome, which hinders process completion.%
\vspace{0.3cm}%
\item%
\textbf{M-13}%
 \ \ \textbf{Model Answer:} B. The commitment to treating writing as a career and consistently producing work, regardless of immediate sales.%
 \\ \textit{Grading Notes:} 3 points for defining the professional mindset based on commitment and consistent output, rather than specific metrics or credentials.%
\vspace{0.3cm}%
\item%
\textbf{M-14}%
 \ \ \textbf{Model Answer:} B. The true value lies in the execution, craft, and revision of the story, not the initial concept.%
 \\ \textit{Grading Notes:} 3 points for understanding the emphasis on execution and craft over the novelty of the initial idea.%
\vspace{0.3cm}%
\item%
\textbf{M-15}%
 \ \ \textbf{Model Answer:} B. They must finish the entire draft before taking extensive critique, as early feedback might derail the organic discovery process.%
 \\ \textit{Grading Notes:} 3 points for recalling the specific warning given to Discovery Writers about protecting their process from premature external influence.%
\vspace{0.3cm}%
\item%
\textbf{S-1}%
 \ \ \textbf{Model Answer:} The Architect is a writer who plans extensively beforehand, structuring the entire story like a blueprint (similar to an Outline Writer). The Gardener is a writer who discovers the story as they write, nurturing it as it grows organically (similar to a Discovery Writer). The instructor notes that most successful writers are not purely one or the other, but rather fall somewhere in the middle of the spectrum, utilizing elements of both planning and discovery depending on the project or stage of writing.%
 \\ \textit{Grading Notes:} 1 pt for defining Architect (planner/outliner). 1 pt for defining Gardener (discoverer/organic). 3 pts for explaining that most writers are a blend/spectrum between the two, often adapting their approach.%
\vspace{0.3cm}%
\item%
\textbf{S-2}%
 \ \ \textbf{Model Answer:} A Cook follows a recipe exactly without understanding the underlying principles or effects of the ingredients/steps. A Chef understands the foundational principles, tools, and techniques of their craft. The desired learning outcome is for students to become 'Chefs'—writers who understand the underlying mechanisms of storytelling (plot, character, setting) so they can intentionally create specific emotional or narrative effects. This mastery is superior for professionals because it allows them to solve unique story problems, adapt to different genres, and move beyond blindly following prescriptive writing advice.%
 \\ \textit{Grading Notes:} 1 pt for defining Cook (follows recipe/prescriptive advice). 1 pt for defining Chef (understands principles/tools/effects). 2 pts for identifying the desired outcome: understanding underlying mechanisms/principles to create intentional effects. 1 pt for explaining why this adaptability is superior for professionals.%
\vspace{0.3cm}%
\item%
\textbf{S-3}%
 \ \ \textbf{Model Answer:} Survivorship Bias is the logical error of focusing only on successful examples (the 'survivors') while ignoring the failures, dropouts, or the role of luck and randomness. In writing, this means overvaluing the specific advice or methods of successful authors (e.g., 'write 2000 words a day') without accounting for the thousands of writers who followed similar advice but failed. The implication is that a new writer might mistakenly believe that replicating a successful author's specific process guarantees success, leading to frustration or adherence to methods that are not suitable for their own style, thus ignoring the high probability that luck played a significant role in the outcome.%
 \\ \textit{Grading Notes:} 2 pts for accurately defining Survivorship Bias (focusing on successes while ignoring failures/luck/randomness). 3 pts for analyzing a specific implication, such as mistakenly believing a successful author's specific process is replicable or guaranteed to work, or failing to recognize the role of luck.%
\vspace{0.3cm}%
\item%
\textbf{S-4}%
 \ \ \textbf{Model Answer:} The professional approach dictates that new writers must prioritize writing multiple complete manuscripts (the 'process') to learn their unique writing style and develop essential skills, particularly revision. This practice is likened to learning scales on a piano. Prioritizing process over immediate sales is crucial because the first few books are primarily practice tools necessary for skill development, not products ready for market. Excessive revision of early works is discouraged because the writer lacks the necessary skill set (the 'revision muscle') to fix the book effectively, and spending too much time on one early manuscript prevents the necessary volume of practice required to become proficient.%
 \\ \textit{Grading Notes:} 1 pt for identifying the priority: writing multiple complete manuscripts/learning the process. 2 pts for explaining the rationale: early books are practice/skill development (like scales), and the writer needs volume to find their unique process. 2 pts for explaining the danger of excessive early revision: lack of necessary revision skills/prevents progress on subsequent, more skilled works.%
\vspace{0.3cm}%
\item%
\textbf{S-5}%
 \ \ \textbf{Model Answer:} The Discovery Writer (or 'Pantser') begins writing with minimal planning, discovering the plot, characters, and structure organically as they go. The Outline Writer (or 'Plotter') meticulously plans the story, often creating detailed chapter-by-chapter outlines before beginning the draft. A challenge for the Discovery Writer is often structural coherence, leading to extensive revision needed to fix plot holes or pacing issues in the later stages. A challenge for the Outline Writer is sometimes losing spontaneity or encountering creative burnout because the story feels too rigid or already 'written' before the drafting begins.%
 \\ \textit{Grading Notes:} 1 pt for defining Discovery Writer (minimal planning/organic discovery). 1 pt for defining Outline Writer (extensive planning/detailed structure). 1.5 pts for identifying a challenge for the Discovery Writer (structural issues, extensive revision needed). 1.5 pts for identifying a challenge for the Outline Writer (loss of spontaneity, rigidity, creative burnout).%
\vspace{0.3cm}%
\item%
\textbf{P-1}%
 \ \ \textbf{Model Answer:} The 'Gardener' and 'Architect' dichotomy describes the foundational *process* or *style* of writing: Architects plan extensively (Outline Writers), while Gardeners discover the story as they write (Discovery Writers). The 'Chef' and 'Cook' metaphor describes the desired *level of mastery* or *understanding* of the craft. A Cook follows a recipe (specific writing advice) without understanding why it works, whereas a Chef understands the underlying tools, ingredients, and processes to create a desired effect.

Regardless of whether a writer is a Gardener or an Architect, the course goal is for them to become a Chef. A Gardener, relying heavily on instinct and discovery, risks producing inconsistent or structurally weak work if they only operate as a Cook (i.e., relying solely on their natural process without understanding the underlying principles of plot, character, and structure). To become a professional Chef, the Gardener must learn the tools of the craft—such as pacing, tension, and revision techniques—so they can consciously analyze and manipulate their discovered material to achieve specific, intentional effects on the reader, moving beyond mere instinct.%
 \\ \textit{Grading Notes:} 10 points total.
2 pts: Accurate definition of Gardener (Discovery Writer) and Architect (Outline Writer) as writing styles/processes.
2 pts: Accurate definition of Cook (follows advice/recipe) and Chef (understands underlying principles/tools) as levels of mastery.
6 pts: Synthesis. Must explain that the Chef status is necessary for both styles, specifically detailing that a Gardener needs the Chef's knowledge to analyze, revise, and intentionally structure the discovered material, ensuring professional quality and coherence rather than relying solely on instinct.%
\vspace{0.3cm}%
\item%
\textbf{P-2}%
 \ \ \textbf{Model Answer:} Survivorship Bias is the logical fallacy where one focuses only on the successful outcomes (the 'survivors') while ignoring the failures or those who dropped out. In writing, this means overvaluing the advice or methods of successful authors without accounting for the role of luck, randomness, or the vast number of writers who followed similar paths but failed.

The instructor's advice to focus on writing multiple books (the first six) mitigates this bias by shifting the focus from external, random validation (sales, luck) to internal, controllable skill acquisition (process refinement). If a writer focuses on completing books, they are prioritizing practice, learning their personal writing process, and mastering revision. This approach acknowledges that early success is rare and often random, ensuring the writer develops the necessary professional skills and resilience to succeed over the long term, regardless of whether their first few attempts 'survive' the market.%
 \\ \textit{Grading Notes:} 10 points total.
3 pts: Accurate definition of Survivorship Bias (focusing only on successful outcomes while ignoring failures/luck).
7 pts: Analysis of the countermeasure. Must explain that focusing on writing multiple books shifts the goal from random external validation (sales) to controllable internal skill development (process, revision, practice). This ensures the writer builds a sustainable career foundation independent of early market luck.%
\vspace{0.3cm}%
\item%
\textbf{P-3}%
 \ \ \textbf{Model Answer:} (1) New writers should avoid over-revising their first chapters because revision is a skill best learned by completing an entire manuscript first. If a writer constantly revises the beginning, they stall their progress and fail to learn the crucial skills required for plotting, pacing, and completing a full narrative arc. The goal of early writing is to finish books to learn the process, treating them like 'learning scales' on a piano, not perfecting the first few notes.

(2) The separation of the 'Artist' and 'Business' personas is critical during the revision and submission phase. The Artist's job is to create, focusing purely on the story and artistic vision without concern for marketability or criticism. Once the manuscript is complete, the Business persona must take over. This persona is objective, unsentimental, and focused on exploiting the work for profit. The Business persona is responsible for critically evaluating the manuscript's flaws (revision), handling rejection, and making necessary changes based on market feedback, ensuring the work is treated as a professional product.%
 \\ \textit{Grading Notes:} 10 points total.
4 pts: Explanation of avoiding over-revision. Must mention that constant revision stalls progress and prevents the writer from learning the full process of completing a narrative (learning the 'scales').
6 pts: Explanation of the Artist/Business split. Must define the Artist's role (creation/vision) and the Business persona's role (objective critique, revision, handling rejection, exploiting the work for profit) specifically during the post-creation phase.%
\vspace{0.3cm}%
\end{enumerate}

%
\end{document}