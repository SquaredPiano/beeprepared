\documentclass[11pt]{article}%
\usepackage[T1]{fontenc}%
\usepackage[utf8]{inputenc}%
\usepackage{lmodern}%
\usepackage{textcomp}%
\usepackage{lastpage}%
\usepackage[margin=1in]{geometry}%
\usepackage{amsmath}%
\usepackage{amssymb}%
\usepackage{titlesec}%
\usepackage{fancyhdr}%
%
%
%
\begin{document}%
\normalsize%
\begin{titlepage}%
\centering%
\vspace*{1cm}%
{\Huge \textbf{ONTARIO TECH UNIVERSITY} \par}%
\vspace{1.5cm}%
{\Large \textbf{Final Exam: Introductory Calculus Lecture 1} \par}%
\vspace{0.5cm}%
{\large \today \par}%
\vspace{2cm}%
\textbf{INSTRUCTIONS TO CANDIDATES} \par%
\vspace{0.5cm}%
\textit{(Formal, precise, and directive. Instructions will emphasize the necessity of showing all steps, clearly labeling variables and functions, and simplifying final answers where appropriate. Tone) Structured Problem Solving (Show All Work). The assessment will consist of multi-part questions requiring the derivation and solution of differential equations and the application of advanced integration techniques reviewed in the lecture. Exam. Please answer all questions.}%
\vspace{3cm}%
\textbf{DO NOT OPEN THIS BOOKLET UNTIL TOLD TO DO SO} \par%
\vfill%
{\large Total Points: 100 \par}%
\end{titlepage}%
\newpage%
\section*{Questions}%
\label{sec:Questions}%
\begin{enumerate}%
\item%
What is the order of the Ordinary Differential Equation (ODE) describing the RLC circuit, given by $$L\frac{d^2Q}{dt^2} + R\frac{dQ}{dt} + \frac{1}{C}Q = V(t)$$? (3 points)%
\begin{itemize}%
\item%
A. Zero-order%
\item%
B. First-order%
\item%
C. Second-order%
\item%
D. Third-order%
\end{itemize}%
\vspace{0.5cm}%
\item%
A differential equation is classified as 'Separable' if it can be written in which of the following general forms? (3 points)%
\begin{itemize}%
\item%
A. $$\frac{dy}{dx} = f(x) + g(y)$$%
\item%
B. $$\frac{dy}{dx} = a(x)b(y)$$%
\item%
C. $$\frac{dy}{dx} + P(x)y = Q(x)$$%
\item%
D. $$\frac{d^2y}{dx^2} + P(x)\frac{dy}{dx} + Q(x)y = 0$$%
\end{itemize}%
\vspace{0.5cm}%
\item%
In the context of the RLC circuit model, the term representing the voltage across the inductor is proportional to the rate of change of current ($$I$$). If $$Q$$ is the charge, which expression correctly represents the inductor voltage? (3 points)%
\begin{itemize}%
\item%
A. $$R I$$%
\item%
B. $$\frac{Q}{C}$$%
\item%
C. $$L \frac{dI}{dt}$$%
\item%
D. $$L \frac{dQ}{dt}$$%
\end{itemize}%
\vspace{0.5cm}%
\item%
Solve the separable differential equation $$\frac{dy}{dx} = x^2 y$$. (3 points)%
\begin{itemize}%
\item%
A. $$y = C e^{2x}$$%
\item%
B. $$y = C e^{x^3/3}$$%
\item%
C. $$y = \frac{x^3}{3} + C$$%
\item%
D. $$y = C x^3$$%
\end{itemize}%
\vspace{0.5cm}%
\item%
The differential equation $$x(y^2-1) + y(x^2-1)\frac{dy}{dx} = 0$$ is separable. Which equation represents the correct separation of variables? (3 points)%
\begin{itemize}%
\item%
A. $$\int \frac{y}{y^2-1} dy = \int \frac{x}{x^2-1} dx$$%
\item%
B. $$\int \frac{y}{y^2-1} dy = -\int \frac{x}{x^2-1} dx$$%
\item%
C. $$\int y(y^2-1) dy = -\int x(x^2-1) dx$$%
\item%
D. $$\int \frac{1}{y(y^2-1)} dy = \int \frac{1}{x(x^2-1)} dx$$%
\end{itemize}%
\vspace{0.5cm}%
\item%
If the rate of radioactive decay ($$N$$ is the number of atoms) is proportional to the number of atoms remaining, which first-order ODE models this process? ($$k$$ is the decay constant, $$k > 0$$) (3 points)%
\begin{itemize}%
\item%
A. $$\frac{dN}{dt} = k N^2$$%
\item%
B. $$\frac{dN}{dt} = k t$$%
\item%
C. $$\frac{dN}{dt} = -k N$$%
\item%
D. $$\frac{dN}{dt} = k$$%
\end{itemize}%
\vspace{0.5cm}%
\item%
The Integration by Parts formula is derived from the product rule of differentiation. If we define $$u = f(x)$$ and $$v = g(x)$$, which statement correctly represents the formula? (3 points)%
\begin{itemize}%
\item%
A. $$\int u dv = uv + \int v du$$%
\item%
B. $$\int u dv = \frac{u}{v} - \int v du$$%
\item%
C. $$\int u dv = uv - \int v du$$%
\item%
D. $$\int u dv = \int v du - uv$$%
\end{itemize}%
\vspace{0.5cm}%
\item%
To evaluate the integral $$\int (2x-1) \ln(x^2+1) dx$$ using Integration by Parts, what is the optimal choice for $$u$$ and $$dv$$? (3 points)%
\begin{itemize}%
\item%
A. $$u = 2x-1, dv = \ln(x^2+1) dx$$%
\item%
B. $$u = \ln(x^2+1), dv = (2x-1) dx$$%
\item%
C. $$u = x^2+1, dv = \frac{2x-1}{x^2+1} dx$$%
\item%
D. $$u = 1, dv = (2x-1) \ln(x^2+1) dx$$%
\end{itemize}%
\vspace{0.5cm}%
\item%
Evaluate the integral $$\int x e^{3x} dx$$. (3 points)%
\begin{itemize}%
\item%
A. $$\frac{1}{3} x e^{3x} - \frac{1}{9} e^{3x} + C$$%
\item%
B. $$x e^{3x} - \frac{1}{3} e^{3x} + C$$%
\item%
C. $$\frac{1}{3} x e^{3x} + \frac{1}{9} e^{3x} + C$$%
\item%
D. $$3 x e^{3x} - 9 e^{3x} + C$$%
\end{itemize}%
\vspace{0.5cm}%
\item%
Evaluate the indefinite integral $$\int \ln(x) dx$$. (3 points)%
\begin{itemize}%
\item%
A. $$\frac{1}{x} + C$$%
\item%
B. $$x \ln(x) - x + C$$%
\item%
C. $$\frac{1}{2} (\ln(x))^2 + C$$%
\item%
D. $$x \ln(x) + x + C$$%
\end{itemize}%
\vspace{0.5cm}%
\item%
Applying Integration by Parts to $$\int x \cos(x) dx$$ results in: (3 points)%
\begin{itemize}%
\item%
A. $$x \sin(x) - \cos(x) + C$$%
\item%
B. $$x \sin(x) + \cos(x) + C$$%
\item%
C. $$\frac{x^2}{2} \sin(x) - \int \frac{x^2}{2} \cos(x) dx$$%
\item%
D. $$-x \cos(x) + \sin(x) + C$$%
\end{itemize}%
\vspace{0.5cm}%
\item%
The reduction formula for $$I_n = \int \cos^n(x) dx$$ is given as $$I_n = \frac{1}{n}\cos^{n-1}(x)\sin(x) + \frac{n-1}{n}I_{n-2}$$. This formula is useful because it: (3 points)%
\begin{itemize}%
\item%
A. Converts the integral into a simpler algebraic equation.%
\item%
B. Reduces the power of the trigonometric function, allowing recursive solution.%
\item%
C. Allows direct substitution using $u = \cos(x)$.%
\item%
D. Only works when $n$ is an odd integer.%
\end{itemize}%
\vspace{0.5cm}%
\item%
To derive the reduction formula for $$I_n = \int \cos^n(x) dx$$, the first step using Integration by Parts requires setting up the integral as $$\int \cos^{n-1}(x) \cdot \cos(x) dx$$. What is the correct choice for $$u$$ and $$dv$$? (3 points)%
\begin{itemize}%
\item%
A. $$u = \cos(x), dv = \cos^{n-1}(x) dx$$%
\item%
B. $$u = \cos^{n-1}(x), dv = \cos(x) dx$$%
\item%
C. $$u = \cos^n(x), dv = dx$$%
\item%
D. $$u = \sin(x), dv = \cos^{n-1}(x) dx$$%
\end{itemize}%
\vspace{0.5cm}%
\item%
A particle of mass $$m$$ moves along the x-axis subject to a force $$F(x, v)$$ that depends on its position ($$x$$) and velocity ($$v = \frac{dx}{dt}$$). According to Newton's Second Law, the resulting differential equation is: (3 points)%
\begin{itemize}%
\item%
A. $$m \frac{dx}{dt} = F(x, v)$$%
\item%
B. $$m \frac{d^2x}{dt^2} = F(x, \frac{dx}{dt})$$%
\item%
C. $$m x = F(x, v)$$%
\item%
D. $$\frac{d^2x}{dt^2} = F(x, v) + m$$%
\end{itemize}%
\vspace{0.5cm}%
\item%
The implicit general solution to the separable equation $$x(y^2-1) + y(x^2-1)\frac{dy}{dx} = 0$$ derived in the lecture review is: (3 points)%
\begin{itemize}%
\item%
A. $$(1-y^2)(1-x^2) = C$$%
\item%
B. $$\ln|y^2-1| + \ln|x^2-1| = C$$%
\item%
C. $$y^2 + x^2 = C$$%
\item%
D. $$y = C \sqrt{1-x^2}$$%
\end{itemize}%
\vspace{0.5cm}%
\item%
Define a **Separable Differential Equation**, including the general form and the method used for its solution. Furthermore, consider the RLC circuit equation derived from Kirchhoff's Law, as discussed in the lecture: $$L\frac{d^2Q}{dt^2} + R\frac{dQ}{dt} + \frac{1}{C}Q = V(t)$$. What is the **order** of this Ordinary Differential Equation (ODE)? (5 points)%
\vspace{4cm}%
\vspace{0.5cm}%
\item%
Evaluate the indefinite integral $$\int (3x^2 + 2x) \ln(x) dx$$ using the method of Integration by Parts. Show all steps clearly, including the choice of $u$ and $dv$. (5 points)%
\vspace{4cm}%
\vspace{0.5cm}%
\item%
The lecture reviewed the reduction formula for $I_n = \int \cos^n(x) dx$. Use the provided formula, $$I_n = \frac{1}{n}\cos^{n-1}(x)\sin(x) + \frac{n-1}{n}I_{n-2}$$, to evaluate the definite integral $$\int_0^{\pi/2} \cos^3(x) dx$$. Show the application of the formula and the evaluation of the resulting terms. (5 points)%
\vspace{4cm}%
\vspace{0.5cm}%
\item%
Find the general solution to the first-order separable differential equation: $$\frac{dy}{dx} = \frac{x^2 y}{y+1}$$ Show all steps of separation and integration, leaving the final answer in implicit form. (5 points)%
\vspace{4cm}%
\vspace{0.5cm}%
\item%
A particle of mass $m$ moves along the x-axis subject to a linear drag force $F(t) = -k v(t)$, where $v(t)$ is the velocity and $k$ is a positive constant. Using Newton's Second Law ($F=ma$), derive the first-order differential equation governing the velocity $v(t)$. Then, solve this ODE to find the general expression for $v(t)$. (5 points)%
\vspace{4cm}%
\vspace{0.5cm}%
\item%
Using the technique of Integration by Parts (IBP), calculate the indefinite integral: 
$$\int x^2 e^{3x} dx$$
Show all steps, including the identification of $u$ and $dv$ for each application of IBP. (10 points)%
\vspace{8cm}%
\vspace{0.5cm}%
\item%
Find the general solution $y(x)$ for the following first-order separable differential equation. Ensure your final answer is expressed implicitly in terms of $y$ and $x$, without explicit integration constants in the denominators or numerators of the final terms.
$$\frac{dy}{dx} = \frac{x \cos(x^2)}{y^3}$$ (10 points)%
\vspace{8cm}%
\vspace{0.5cm}%
\item%
The decay of a radioactive substance is modeled by the first-order ordinary differential equation (ODE):
$$\frac{dN}{dt} = -k N$$
where $N(t)$ is the amount of the substance remaining at time $t$, and $k$ is a positive decay constant.

(a) Solve this separable differential equation to find the general solution $N(t)$. (6 points)
(b) If the initial amount of the substance is $N_0$ (i.e., $N(0) = N_0$), find the particular solution. (4 points) (10 points)%
\vspace{8cm}%
\vspace{0.5cm}%
\end{enumerate}

%
\newpage%
\section*{Solution Key & Grading Rubric}%
\label{sec:SolutionKeyGradingRubric}%
\textbf{Confidential - Instructor Use Only}%
\vspace{0.5cm}%
\begin{enumerate}%
\item%
\textbf{M-1}%
 \ \ \textbf{Model Answer:} C. Second-order%
 \\ \textit{Grading Notes:} 1 pt for recognizing the definition of ODE order. 2 pts for correctly identifying the highest derivative as the second derivative ($\frac{d^2Q}{dt^2}$).%
\vspace{0.3cm}%
\item%
\textbf{M-2}%
 \ \ \textbf{Model Answer:} B. $$\frac{dy}{dx} = a(x)b(y)$$%
 \\ \textit{Grading Notes:} 3 pts for correct recall of the definition of a separable equation, where the derivative can be expressed as a product of a function of $x$ and a function of $y$.%
\vspace{0.3cm}%
\item%
\textbf{M-3}%
 \ \ \textbf{Model Answer:} C. $$L \frac{dI}{dt}$$%
 \\ \textit{Grading Notes:} 1 pt for identifying the inductor voltage term. 2 pts for recalling the correct physical relationship ($$V_L = L \frac{dI}{dt}$$).%
\vspace{0.3cm}%
\item%
\textbf{M-4}%
 \ \ \textbf{Model Answer:} B. $$y = C e^{x^3/3}$$%
 \\ \textit{Grading Notes:} 1 pt for separating variables: $$\int \frac{1}{y} dy = \int x^2 dx$$. 1 pt for correct integration: $$\ln|y| = \frac{x^3}{3} + K$$. 1 pt for solving for $y$ and correctly handling the constant: $$y = C e^{x^3/3}$$.%
\vspace{0.3cm}%
\item%
\textbf{M-5}%
 \ \ \textbf{Model Answer:} B. $$\int \frac{y}{y^2-1} dy = -\int \frac{x}{x^2-1} dx$$%
 \\ \textit{Grading Notes:} 1 pt for isolating the derivative term: $$y(x^2-1)\frac{dy}{dx} = -x(y^2-1)$$. 1 pt for correctly grouping $x$ and $y$ terms: $$\frac{y}{y^2-1} dy = -\frac{x}{x^2-1} dx$$. 1 pt for setting up the integral correctly.%
\vspace{0.3cm}%
\item%
\textbf{M-6}%
 \ \ \textbf{Model Answer:} C. $$\frac{dN}{dt} = -k N$$%
 \\ \textit{Grading Notes:} 1 pt for understanding that 'rate of change' implies $$\frac{dN}{dt}$$. 1 pt for understanding 'proportional to N' implies $$k N$$. 1 pt for correctly including the negative sign, indicating decay (decrease).%
\vspace{0.3cm}%
\item%
\textbf{M-7}%
 \ \ \textbf{Model Answer:} C. $$\int u dv = uv - \int v du$$%
 \\ \textit{Grading Notes:} 3 pts for accurate recall of the standard Integration by Parts formula.%
\vspace{0.3cm}%
\item%
\textbf{M-8}%
 \ \ \textbf{Model Answer:} B. $$u = \ln(x^2+1), dv = (2x-1) dx$$%
 \\ \textit{Grading Notes:} 1 pt for recognizing the need for Integration by Parts. 2 pts for applying the LIATE rule, choosing $$u$$ as the logarithmic term (which simplifies upon differentiation) and $$dv$$ as the polynomial term (which is easily integrable).%
\vspace{0.3cm}%
\item%
\textbf{M-9}%
 \ \ \textbf{Model Answer:} A. $$\frac{1}{3} x e^{3x} - \frac{1}{9} e^{3x} + C$$%
 \\ \textit{Grading Notes:} 1 pt for correct setup: $$u=x, dv=e^{3x}dx \implies du=dx, v=\frac{1}{3}e^{3x}$$. 1 pt for applying the formula: $$\frac{1}{3}xe^{3x} - \int \frac{1}{3}e^{3x} dx$$. 1 pt for final correct integration and constant: $$\frac{1}{3}xe^{3x} - \frac{1}{9}e^{3x} + C$$.%
\vspace{0.3cm}%
\item%
\textbf{M-10}%
 \ \ \textbf{Model Answer:} B. $$x \ln(x) - x + C$$%
 \\ \textit{Grading Notes:} 1 pt for recognizing the need for Integration by Parts with $$u = \ln(x)$$ and $$dv = dx$$. 1 pt for finding $$du = \frac{1}{x} dx$$ and $$v = x$$. 1 pt for correct application and final result: $$x \ln(x) - \int x \cdot \frac{1}{x} dx = x \ln(x) - x + C$$.%
\vspace{0.3cm}%
\item%
\textbf{M-11}%
 \ \ \textbf{Model Answer:} B. $$x \sin(x) + \cos(x) + C$$%
 \\ \textit{Grading Notes:} 1 pt for setup: $$u=x, dv=\cos(x)dx \implies du=dx, v=\sin(x)$$. 1 pt for applying the formula: $$x \sin(x) - \int \sin(x) dx$$. 1 pt for correct final integration: $$x \sin(x) - (-\cos(x)) + C = x \sin(x) + \cos(x) + C$$.%
\vspace{0.3cm}%
\item%
\textbf{M-12}%
 \ \ \textbf{Model Answer:} B. Reduces the power of the trigonometric function, allowing recursive solution.%
 \\ \textit{Grading Notes:} 3 pts for understanding the purpose of a reduction formula: to express an integral in terms of a similar integral of a lower order (or power), enabling recursive calculation until a known base integral is reached.%
\vspace{0.3cm}%
\item%
\textbf{M-13}%
 \ \ \textbf{Model Answer:} B. $$u = \cos^{n-1}(x), dv = \cos(x) dx$$%
 \\ \textit{Grading Notes:} 1 pt for splitting the term. 2 pts for correctly choosing $$u$$ as the term whose derivative is manageable (using the chain rule) and $$dv$$ as the easily integrable term, which is crucial for isolating $I_n$ later in the derivation.%
\vspace{0.3cm}%
\item%
\textbf{M-14}%
 \ \ \textbf{Model Answer:} B. $$m \frac{d^2x}{dt^2} = F(x, \frac{dx}{dt})$$%
 \\ \textit{Grading Notes:} 1 pt for recalling Newton's Second Law ($$F=ma$$). 2 pts for correctly expressing acceleration ($$a$$) as the second derivative of position ($$x$$) with respect to time ($$t$$): $$a = \frac{d^2x}{dt^2}$$.%
\vspace{0.3cm}%
\item%
\textbf{M-15}%
 \ \ \textbf{Model Answer:} A. $$(1-y^2)(1-x^2) = C$$%
 \\ \textit{Grading Notes:} 1 pt for correct separation: $$\frac{y}{y^2-1} dy = -\frac{x}{x^2-1} dx$$. 1 pt for correct integration: $$\frac{1}{2}\ln|y^2-1| = -\frac{1}{2}\ln|x^2-1| + K$$. 1 pt for correct algebraic simplification leading to the product form: $$(y^2-1)(x^2-1) = C_1$$, which is equivalent to $$(1-y^2)(1-x^2) = C$$.%
\vspace{0.3cm}%
\item%
\textbf{S-1}%
 \ \ \textbf{Model Answer:} A Separable Differential Equation is a first-order ODE that can be written in the form $$\frac{dy}{dx} = a(x)b(y)$$. The solution method involves separating the variables and integrating each side independently: $$\int \frac{1}{b(y)} dy = \int a(x) dx$$. The order of the RLC circuit ODE, $$L\frac{d^2Q}{dt^2} + R\frac{dQ}{dt} + \frac{1}{C}Q = V(t)$$, is **2**, because the highest derivative present is the second derivative, $\frac{d^2Q}{dt^2}$.%
 \\ \textit{Grading Notes:} 2 pts for correct definition of Separable Equation (must include the form $a(x)b(y)$ and the separation method). 3 pts for correctly identifying the order as 2 and justifying it based on the highest derivative.%
\vspace{0.3cm}%
\item%
\textbf{S-2}%
 \ \ \textbf{Model Answer:} We use Integration by Parts: $\int u dv = uv - \int v du$.
1. **Choose u and dv:** Let $u = \ln(x)$ and $dv = (3x^2 + 2x) dx$.
2. **Calculate du and v:** $du = \frac{1}{x} dx$ and $v = \int (3x^2 + 2x) dx = x^3 + x^2$.
3. **Apply IBP formula:**
$$\int (3x^2 + 2x) \ln(x) dx = \ln(x)(x^3 + x^2) - \int (x^3 + x^2) \left(\frac{1}{x}\right) dx$$
4. **Simplify and Integrate:**
$$= (x^3 + x^2)\ln(x) - \int (x^2 + x) dx$$
$$= (x^3 + x^2)\ln(x) - \left(\frac{x^3}{3} + \frac{x^2}{2}\right) + C$$%
 \\ \textit{Grading Notes:} 1 pt for correct choice of $u$ and $dv$. 1 pt for correctly calculating $du$ and $v$. 1 pt for correct application of the IBP formula. 2 pts for correctly integrating the resulting polynomial integral and stating the final answer with $+C$.%
\vspace{0.3cm}%
\item%
\textbf{S-3}%
 \ \ \textbf{Model Answer:} We need to calculate $I_3 = \int_0^{\pi/2} \cos^3(x) dx$.
1. **Apply Reduction Formula (n=3):**
$$I_3 = \left[ \frac{1}{3}\cos^{2}(x)\sin(x) \right]_0^{\pi/2} + \frac{2}{3}I_{1}$$
2. **Evaluate Boundary Term:**
$$\left[ \frac{1}{3}\cos^{2}(x)\sin(x) \right]_0^{\pi/2} = \left( \frac{1}{3}\cos^{2}(\pi/2)\sin(\pi/2) \right) - \left( \frac{1}{3}\cos^{2}(0)\sin(0) \right)$$
$$= \left( \frac{1}{3}(0)(1) \right) - \left( \frac{1}{3}(1)(0) \right) = 0$$
3. **Calculate $I_1$:**
$$I_1 = \int_0^{\pi/2} \cos(x) dx = [\sin(x)]_0^{\pi/2} = \sin(\pi/2) - \sin(0) = 1 - 0 = 1$$
4. **Final Result:**
$$I_3 = 0 + \frac{2}{3}(1) = \frac{2}{3}$$%
 \\ \textit{Grading Notes:} 1 pt for correctly setting up the reduction formula for $n=3$. 2 pts for correctly evaluating the boundary term to zero. 1 pt for correctly calculating $I_1 = 1$. 1 pt for the final correct numerical answer $\frac{2}{3}$.%
\vspace{0.3cm}%
\item%
\textbf{S-4}%
 \ \ \textbf{Model Answer:} 1. **Separate the variables:**
$$\frac{y+1}{y} dy = x^2 dx$$
2. **Simplify and Integrate both sides:**
$$\int \left(1 + \frac{1}{y}\right) dy = \int x^2 dx$$
3. **Perform integration:**
$$y + \ln|y| = \frac{x^3}{3} + C$$%
 \\ \textit{Grading Notes:} 1 pt for correctly separating the variables. 1 pt for correctly simplifying the LHS integral expression $\int (1 + 1/y) dy$. 2 pts for correctly integrating both sides ($y + \ln|y|$ and $x^3/3$). 1 pt for including the constant of integration $C$.%
\vspace{0.3cm}%
\item%
\textbf{S-5}%
 \ \ \textbf{Model Answer:} 1. **Derivation of ODE:** Using Newton's Second Law $F = ma$, and substituting $a = \frac{dv}{dt}$ and $F = -kv$:
$$m \frac{dv}{dt} = -kv$$
2. **Separation of Variables:**
$$\frac{dv}{v} = -\frac{k}{m} dt$$
3. **Integration:**
$$\int \frac{1}{v} dv = \int -\frac{k}{m} dt$$
$$\ln|v| = -\frac{k}{m} t + C_1$$
4. **General Solution for v(t):** Exponentiating both sides:
$$|v| = e^{-kt/m + C_1} = e^{C_1} e^{-kt/m}$$
Let $C = \pm e^{C_1}$ (arbitrary constant):
$$v(t) = C e^{-\frac{k}{m} t}$$%
 \\ \textit{Grading Notes:} 1 pt for correctly setting up the ODE $m \frac{dv}{dt} = -kv$. 1 pt for correctly separating the variables. 2 pts for correct integration leading to $\ln|v| = -\frac{k}{m} t + C_1$. 1 pt for solving explicitly for $v(t)$ as $v(t) = C e^{-\frac{k}{m} t}$.%
\vspace{0.3cm}%
\item%
\textbf{P-1}%
 \ \ \textbf{Model Answer:} We apply Integration by Parts, $\int u dv = uv - \int v du$, twice.

**First Application:**
Let $u_1 = x^2$ and $dv_1 = e^{3x} dx$.
Then $du_1 = 2x dx$ and $v_1 = \frac{1}{3}e^{3x}$.

$$\int x^2 e^{3x} dx = \frac{1}{3}x^2 e^{3x} - \int \frac{1}{3}e^{3x} (2x) dx$$
$$\int x^2 e^{3x} dx = \frac{1}{3}x^2 e^{3x} - \frac{2}{3}\int x e^{3x} dx$$ 

**Second Application (on $\int x e^{3x} dx$):**
Let $u_2 = x$ and $dv_2 = e^{3x} dx$.
Then $du_2 = dx$ and $v_2 = \frac{1}{3}e^{3x}$.

$$\int x e^{3x} dx = \frac{1}{3}x e^{3x} - \int \frac{1}{3}e^{3x} dx$$
$$\int x e^{3x} dx = \frac{1}{3}x e^{3x} - \frac{1}{9}e^{3x}$$ 

**Final Solution:**
Substitute the result of the second integral back into the first expression:
$$\int x^2 e^{3x} dx = \frac{1}{3}x^2 e^{3x} - \frac{2}{3}\left[ \frac{1}{3}x e^{3x} - \frac{1}{9}e^{3x} \right] + C$$
$$\int x^2 e^{3x} dx = \frac{1}{3}x^2 e^{3x} - \frac{2}{9}x e^{3x} + \frac{2}{27}e^{3x} + C$$%
 \\ \textit{Grading Notes:} 1 pt: Correct identification of $u_1$ and $dv_1$.
3 pts: Correct execution of the first IBP step, resulting in $\frac{1}{3}x^2 e^{3x} - \frac{2}{3}\int x e^{3x} dx$.
3 pts: Correct execution of the second IBP step, solving $\int x e^{3x} dx$.
2 pts: Correct substitution and algebraic simplification of the final expression.
1 pt: Inclusion of the constant of integration, $C$. (Total 10 pts)%
\vspace{0.3cm}%
\item%
\textbf{P-2}%
 \ \ \textbf{Model Answer:} The equation is separable. We rearrange the terms to separate $y$ and $x$:
$$y^3 dy = x \cos(x^2) dx$$

Integrate both sides:
$$\int y^3 dy = \int x \cos(x^2) dx$$

**Left Hand Side (LHS):**
$$\int y^3 dy = \frac{y^4}{4} + C_1$$

**Right Hand Side (RHS):**
We use substitution for the RHS. Let $u = x^2$, so $du = 2x dx$, or $x dx = \frac{1}{2} du$.
$$\int x \cos(x^2) dx = \int \cos(u) \frac{1}{2} du = \frac{1}{2} \sin(u) + C_2$$
$$\int x \cos(x^2) dx = \frac{1}{2} \sin(x^2) + C_2$$

**General Solution:**
Equating LHS and RHS:
$$\frac{y^4}{4} + C_1 = \frac{1}{2} \sin(x^2) + C_2$$
$$\frac{y^4}{4} = \frac{1}{2} \sin(x^2) + (C_2 - C_1)$$
Let $C = C_2 - C_1$. Multiply the entire equation by 4 to absorb the fractions into the constant:
$$y^4 = 2 \sin(x^2) + 4C$$
Let $K = 4C$. The general solution is:
$$y^4 = 2 \sin(x^2) + K$$%
 \\ \textit{Grading Notes:} 2 pts: Correctly separating the variables: $y^3 dy = x \cos(x^2) dx$.
2 pts: Correct integration of the LHS: $\int y^3 dy = y^4/4$.
3 pts: Correct use of substitution ($u=x^2$) and integration of the RHS: $\frac{1}{2} \sin(x^2)$.
2 pts: Combining the results and correctly handling the integration constant.
1 pt: Final algebraic simplification to the form $y^4 = 2 \sin(x^2) + K$. (Total 10 pts)%
\vspace{0.3cm}%
\item%
\textbf{P-3}%
 \ \ \textbf{Model Answer:} (a) General Solution:

1. Separate the variables $N$ and $t$:
$$\frac{dN}{N} = -k dt$$

2. Integrate both sides:
$$\int \frac{1}{N} dN = \int -k dt$$
$$\ln|N| = -kt + C_1$$

3. Solve for $N(t)$:
$$|N| = e^{-kt + C_1} = e^{C_1} e^{-kt}$$
Since $N$ represents an amount, $N > 0$. Let $C = e^{C_1}$ (a positive constant).
$$N(t) = C e^{-kt}$$

(b) Particular Solution:

1. Apply the initial condition $N(0) = N_0$ to the general solution $N(t) = C e^{-kt}$:
$$N_0 = C e^{-k(0)}$$
$$N_0 = C (1)$$
$$C = N_0$$

2. The particular solution is:
$$N(t) = N_0 e^{-kt}$$%
 \\ \textit{Grading Notes:} (a) General Solution (6 pts):
2 pts: Correct separation of variables $\frac{dN}{N} = -k dt$.
2 pts: Correct integration leading to $\ln|N| = -kt + C_1$.
2 pts: Correctly solving for $N(t)$ in the form $N(t) = C e^{-kt}$.

(b) Particular Solution (4 pts):
2 pts: Correctly applying the initial condition $N(0) = N_0$.
2 pts: Correctly determining the constant $C=N_0$ and stating the final particular solution $N(t) = N_0 e^{-kt}$. (Total 10 pts)%
\vspace{0.3cm}%
\end{enumerate}

%
\end{document}