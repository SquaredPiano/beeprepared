\documentclass[11pt]{article}%
\usepackage[T1]{fontenc}%
\usepackage[utf8]{inputenc}%
\usepackage{lmodern}%
\usepackage{textcomp}%
\usepackage{lastpage}%
\usepackage[margin=1in]{geometry}%
\usepackage{amsmath}%
\usepackage{amssymb}%
\usepackage{titlesec}%
\usepackage{fancyhdr}%
%
%
%
\begin{document}%
\normalsize%
\begin{titlepage}%
\centering%
\vspace*{1cm}%
{\Huge \textbf{ONTARIO TECH UNIVERSITY} \par}%
\vspace{1.5cm}%
{\Large \textbf{Final Exam: Introduction to Stoicism} \par}%
\vspace{0.5cm}%
{\large \today \par}%
\vspace{2cm}%
\textbf{INSTRUCTIONS TO CANDIDATES} \par%
\vspace{0.5cm}%
\textit{(Formal, rigorous, and encouraging. Instructions must clearly delineate expectations for depth of analysis and citation of specific Stoic principles (e.g., 'Analyze the scenario using the dichotomy of control as articulated by Epictetus'). Tone) Mixed format assessment combining definitional recall (matching/short answer) with short-form essay questions requiring conceptual application and comparative analysis of core Stoic tenets (e.g., comparing the role of Logos vs. the Cardinal Virtues). Exam. Please answer all questions.}%
\vspace{3cm}%
\textbf{DO NOT OPEN THIS BOOKLET UNTIL TOLD TO DO SO} \par%
\vfill%
{\large Total Points: 100 \par}%
\end{titlepage}%
\newpage%
\section*{Questions}%
\label{sec:Questions}%
\begin{enumerate}%
\item%
According to Stoic cosmology, what term refers to the rational, governing principle of the universe, often understood as fate or divine reason? (3 points)%
\begin{itemize}%
\item%
A. Hedone%
\item%
B. Pathos%
\item%
C. Logos%
\item%
D. Eudaimonia%
\end{itemize}%
\vspace{0.5cm}%
\item%
The Stoic goal of achieving inner tranquility and freedom from disturbance, often compared to the Buddhist concept of Nirvana, is known as: (3 points)%
\begin{itemize}%
\item%
A. Apatheia%
\item%
B. Ataraxia%
\item%
C. Arete%
\item%
D. Phronesis%
\end{itemize}%
\vspace{0.5cm}%
\item%
Which of the four cardinal virtues is defined by the ability to navigate complex situations in a logical, informed, and calm manner, often referred to as 'moral knowledge'? (3 points)%
\begin{itemize}%
\item%
A. Courage (Andreia)%
\item%
B. Justice (Dikaiosyne)%
\item%
C. Temperance (Sophrosyne)%
\item%
D. Practical Wisdom (Phronesis)%
\end{itemize}%
\vspace{0.5cm}%
\item%
A Stoic is faced with a sudden, unexpected financial loss. Applying the Dichotomy of Control, what is the *only* factor they can control? (3 points)%
\begin{itemize}%
\item%
A. The speed of recovering the lost funds.%
\item%
B. The external event of the loss itself.%
\item%
C. Their judgment and response to the loss.%
\item%
D. The opinions of others regarding their misfortune.%
\end{itemize}%
\vspace{0.5cm}%
\item%
The Stoic virtue of Justice (Dikaiosyne) is unique because it requires the practitioner to extend fairness and humane treatment even towards: (3 points)%
\begin{itemize}%
\item%
A. Preferred Indifferents.%
\item%
B. Those who have caused them harm or injustice.%
\item%
C. The governing Logos of the universe.%
\item%
D. Their own internal passions (pathē).%
\end{itemize}%
\vspace{0.5cm}%
\item%
How does the virtue of Temperance (Sophrosyne) relate to 'preferred indifferents' (e.g., health, wealth)? (3 points)%
\begin{itemize}%
\item%
A. Temperance requires the complete rejection and avoidance of all preferred indifferents.%
\item%
B. Temperance ensures that one pursues preferred indifferents only with moderation and without emotional attachment.%
\item%
C. Temperance dictates that preferred indifferents must be used to achieve Ataraxia directly.%
\item%
D. Temperance is irrelevant to indifferents, applying only to moral actions.%
\end{itemize}%
\vspace{0.5cm}%
\item%
Epictetus famously stated, 'Men are disturbed not by things, but by the views which they take of them.' This statement is the foundational principle for which modern psychological therapy? (3 points)%
\begin{itemize}%
\item%
A. Psychoanalysis%
\item%
B. Rational Emotive Behavioral Therapy (REBT)%
\item%
C. Cognitive Behavioral Therapy (CBT)%
\item%
D. Humanistic Psychology%
\end{itemize}%
\vspace{0.5cm}%
\item%
Which historical figure, known as the 'Stoic Emperor,' utilized Stoic principles to manage the immense pressures of leading the Roman Empire through wars and personal loss, documenting his reflections in *Meditations*? (3 points)%
\begin{itemize}%
\item%
A. Zeno of Cyprus%
\item%
B. Seneca%
\item%
C. Epictetus%
\item%
D. Marcus Aurelius%
\end{itemize}%
\vspace{0.5cm}%
\item%
The Stoic concept of Courage (Andreia) primarily involves: (3 points)%
\begin{itemize}%
\item%
A. Physical bravery in battle.%
\item%
B. The willingness to face daily challenges and adversity with clarity and integrity.%
\item%
C. The ability to suppress all feelings of fear.%
\item%
D. Accepting the inevitability of death without reflection.%
\end{itemize}%
\vspace{0.5cm}%
\item%
If a Stoic accepts the interconnected web of cause and effect (Logos), this acceptance most directly supports the practice of which virtue? (3 points)%
\begin{itemize}%
\item%
A. Temperance (moderation of desires)%
\item%
B. Courage (facing inevitable hardship)%
\item%
C. Justice (fairness to others)%
\item%
D. Practical Wisdom (logical decision-making)%
\end{itemize}%
\vspace{0.5cm}%
\item%
Which statement accurately reflects the Stoic view on 'pathē' (destructive emotions like fear, excessive desire, or anger)? (3 points)%
\begin{itemize}%
\item%
A. Pathē are natural, unavoidable human reactions that must be managed.%
\item%
B. Pathē are errors in judgment about things outside our control.%
\item%
C. Pathē are necessary catalysts for virtuous action.%
\item%
D. Pathē are external forces imposed by fate (Logos).%
\end{itemize}%
\vspace{0.5cm}%
\item%
Zeno of Cyprus founded Stoicism after a significant life event led him to philosophy. What was this event? (3 points)%
\begin{itemize}%
\item%
A. He was exiled from Athens for political reasons.%
\item%
B. He lost his wealth in a shipwreck near Athens.%
\item%
C. He inherited a large sum of money and decided to dedicate his life to study.%
\item%
D. He served as a general in the Macedonian army.%
\end{itemize}%
\vspace{0.5cm}%
\item%
The Stoic emphasis on finding meaning and purpose, even in suffering, strongly influenced Viktor Frankl's therapeutic approach. This approach is known as: (3 points)%
\begin{itemize}%
\item%
A. Existentialism%
\item%
B. Logotherapy%
\item%
C. Psychoanalysis%
\item%
D. Behaviorism%
\end{itemize}%
\vspace{0.5cm}%
\item%
When Seneca advocated for the humane treatment of slaves, he was primarily demonstrating the application of which cardinal virtue? (3 points)%
\begin{itemize}%
\item%
A. Temperance%
\item%
B. Practical Wisdom%
\item%
C. Justice%
\item%
D. Courage%
\end{itemize}%
\vspace{0.5cm}%
\item%
The core difference between the Stoic concept of Virtue (Arete) and 'preferred indifferents' (like health or reputation) is that Virtue is the sole good, while preferred indifferents are: (3 points)%
\begin{itemize}%
\item%
A. Necessary for happiness but morally neutral.%
\item%
B. Morally bad and must be avoided entirely.%
\item%
C. Things that contribute to well-being but have no intrinsic moral value.%
\item%
D. External factors that are entirely within our control.%
\end{itemize}%
\vspace{0.5cm}%
\item%
Define the Stoic concept of **Logos**. Explain how understanding the Logos informs the Stoic pursuit of virtue and the goal of achieving *Ataraxia* (tranquility). (5 points)%
\vspace{4cm}%
\vspace{0.5cm}%
\item%
The Stoics argued that the four Cardinal Virtues (Practical Wisdom, Temperance, Justice, and Courage) are inseparable. Analyze the relationship between **Practical Wisdom** and **Courage**. Why is mere physical bravery insufficient without Practical Wisdom? (5 points)%
\vspace{4cm}%
\vspace{0.5cm}%
\item%
A student receives a failing grade on a major exam despite studying diligently. Apply the Stoic **Dichotomy of Control** to this scenario. Identify one factor that is internal (within their control) and one factor that is external (outside their control), and explain the appropriate Stoic response to the external factor. (5 points)%
\vspace{4cm}%
\vspace{0.5cm}%
\item%
Compare the practical application of Stoicism by **Marcus Aurelius** (Emperor) and **Seneca** (Advisor/Writer). Given their distinct political and social contexts, how might their emphasis on the virtue of **Justice** differ in scope or focus? (5 points)%
\vspace{4cm}%
\vspace{0.5cm}%
\item%
Epictetus stated that "Men are disturbed not by things, but by the views which they take of them." Explain how this core Stoic principle forms the foundational premise for modern cognitive therapies, specifically **Rational Emotive Behavioral Therapy (REBT)**. (5 points)%
\vspace{4cm}%
\vspace{0.5cm}%
\item%
Analyze the relationship between the Stoic concept of Logos (the rational structure of the universe) and the four Cardinal Virtues (Practical Wisdom, Temperance, Justice, and Courage). Specifically, explain how the acceptance of Logos provides the necessary foundation for the consistent practice of both Courage and Justice, even in the face of personal adversity. (10 points)%
\vspace{8cm}%
\vspace{0.5cm}%
\item%
A junior colleague, whom you mentored, publicly plagiarizes a key section of your research paper, leading to your immediate professional censure and the loss of a major grant opportunity (an external event). 

Using the Stoic Dichotomy of Control, analyze this scenario. Identify the specific Cardinal Virtue most crucial for responding correctly to this betrayal and professional setback, and explain the steps a Stoic would take to apply that virtue to maintain inner peace (Ataraxia). (10 points)%
\vspace{8cm}%
\vspace{0.5cm}%
\item%
Define the Stoic goal of $\text{Ataraxia}$ and the metaphysical concept of $\text{Logos}$. Using the teachings of Epictetus (specifically, the idea that suffering stems from judgments, not events), synthesize how a Stoic sage utilizes their understanding of $\text{Logos}$ to systematically eliminate the emotional disturbances (pathē) that prevent the achievement of $\text{Ataraxia}$. (10 points)%
\vspace{8cm}%
\vspace{0.5cm}%
\end{enumerate}

%
\newpage%
\section*{Solution Key & Grading Rubric}%
\label{sec:SolutionKeyGradingRubric}%
\textbf{Confidential - Instructor Use Only}%
\vspace{0.5cm}%
\begin{enumerate}%
\item%
\textbf{M-1}%
 \ \ \textbf{Model Answer:} C. Logos%
 \\ \textit{Grading Notes:} 3 points for correctly identifying Logos as the rational structure of the universe. 1 point for recognizing it is a core Stoic term, but confusing it with Eudaimonia (flourishing).%
\vspace{0.3cm}%
\item%
\textbf{M-2}%
 \ \ \textbf{Model Answer:} B. Ataraxia%
 \\ \textit{Grading Notes:} 3 points for correctly defining Ataraxia (tranquility of mind). 2 points if the student selects Apatheia (freedom from destructive passions), showing understanding of the emotional goal but missing the broader state of tranquility.%
\vspace{0.3cm}%
\item%
\textbf{M-3}%
 \ \ \textbf{Model Answer:} D. Practical Wisdom (Phronesis)%
 \\ \textit{Grading Notes:} 3 points for accurately identifying Practical Wisdom (Phronesis). 1 point for selecting Temperance, demonstrating knowledge of the virtues but confusing the specific function.%
\vspace{0.3cm}%
\item%
\textbf{M-4}%
 \ \ \textbf{Model Answer:} C. Their judgment and response to the loss.%
 \\ \textit{Grading Notes:} 3 points for correctly applying the Dichotomy of Control, recognizing that internal judgments are the only controllable factor. 1 point if they select A or D, showing they understand the situation is external but failing to pinpoint the internal response.%
\vspace{0.3cm}%
\item%
\textbf{M-5}%
 \ \ \textbf{Model Answer:} B. Those who have caused them harm or injustice.%
 \\ \textit{Grading Notes:} 3 points for understanding the demanding social application of Justice, particularly its universal scope (as exemplified by Seneca's views on slaves). 1 point for selecting D, confusing social virtue with internal self-control.%
\vspace{0.3cm}%
\item%
\textbf{M-6}%
 \ \ \textbf{Model Answer:} B. Temperance ensures that one pursues preferred indifferents only with moderation and without emotional attachment.%
 \\ \textit{Grading Notes:} 3 points for synthesizing the concepts: Temperance governs the appropriate, non-addictive use of things that are morally neutral but naturally desirable. 1 point for selecting A, demonstrating knowledge of detachment but misunderstanding the Stoic allowance for 'preferred' items.%
\vspace{0.3cm}%
\item%
\textbf{M-7}%
 \ \ \textbf{Model Answer:} B. Rational Emotive Behavioral Therapy (REBT)%
 \\ \textit{Grading Notes:} 3 points for correctly linking Epictetus's core insight to REBT, which directly addresses irrational beliefs (judgments). 2 points for selecting CBT, which is closely related but less directly cited as having Stoic origins than REBT.%
\vspace{0.3cm}%
\item%
\textbf{M-8}%
 \ \ \textbf{Model Answer:} D. Marcus Aurelius%
 \\ \textit{Grading Notes:} 3 points for correctly identifying Marcus Aurelius and his role. 1 point for selecting Seneca, demonstrating knowledge of Roman Stoics but confusing the figures.%
\vspace{0.3cm}%
\item%
\textbf{M-9}%
 \ \ \textbf{Model Answer:} B. The willingness to face daily challenges and adversity with clarity and integrity.%
 \\ \textit{Grading Notes:} 3 points for recognizing that Stoic Courage is a moral and intellectual virtue applied to daily life, not just physical heroism. 1 point for selecting A, demonstrating a common but incomplete understanding of the term.%
\vspace{0.3cm}%
\item%
\textbf{M-10}%
 \ \ \textbf{Model Answer:} B. Courage (facing inevitable hardship)%
 \\ \textit{Grading Notes:} 3 points for synthesizing the concepts: Accepting Logos means accepting fate, which requires Courage to endure inevitable hardships without complaint. 2 points for selecting Practical Wisdom, as understanding Logos aids decision-making, but Courage is the direct emotional response required by acceptance.%
\vspace{0.3cm}%
\item%
\textbf{M-11}%
 \ \ \textbf{Model Answer:} B. Pathē are errors in judgment about things outside our control.%
 \\ \textit{Grading Notes:} 3 points for precision: Stoics viewed destructive emotions not as mere feelings but as cognitive errors (false judgments). 1 point for selecting A, demonstrating a modern, but not strictly Stoic, understanding of emotion.%
\vspace{0.3cm}%
\item%
\textbf{M-12}%
 \ \ \textbf{Model Answer:} B. He lost his wealth in a shipwreck near Athens.%
 \\ \textit{Grading Notes:} 3 points for recalling the specific historical origin story of Zeno. 1 point for selecting A, demonstrating general knowledge of ancient Greek philosophical context.%
\vspace{0.3cm}%
\item%
\textbf{M-13}%
 \ \ \textbf{Model Answer:} B. Logotherapy%
 \\ \textit{Grading Notes:} 3 points for correctly linking the Stoic search for meaning (Logos/Virtue) to Frankl's Logotherapy. 1 point for selecting Existentialism, which shares thematic overlap but is not the specific therapeutic model.%
\vspace{0.3cm}%
\item%
\textbf{M-14}%
 \ \ \textbf{Model Answer:} C. Justice%
 \\ \textit{Grading Notes:} 3 points for recognizing that the humane treatment of others, regardless of social status, falls under the domain of Justice (fairness and duty to humanity). 1 point for selecting Practical Wisdom, as it informs the decision, but Justice is the ethical action.%
\vspace{0.3cm}%
\item%
\textbf{M-15}%
 \ \ \textbf{Model Answer:} C. Things that contribute to well-being but have no intrinsic moral value.%
 \\ \textit{Grading Notes:} 3 points for accurately defining preferred indifferents as morally neutral but naturally advantageous (lacking intrinsic moral value). 2 points for selecting A, demonstrating an understanding of their necessity for happiness but missing the crucial distinction of moral neutrality.%
\vspace{0.3cm}%
\item%
\textbf{S-1}%
 \ \ \textbf{Model Answer:} Logos is the Stoic concept of the rational structure of the universe—the interconnected web of cause and effect that governs all events. It represents divine reason or universal law. Understanding Logos means accepting that the universe operates according to a necessary, rational order. The pursuit of virtue is the only appropriate response to this order, as virtue (living rationally) is synonymous with living 'in accordance with nature.' By aligning one's will with the Logos and accepting external events as necessary, the Stoic eliminates irrational desires and fears, thereby achieving *Ataraxia* (tranquility of mind).%
 \\ \textit{Grading Notes:} 1 pt for accurately defining Logos as universal reason/rational structure. 2 pts for explaining that virtue is the necessary alignment with this rational structure ('living in accordance with nature'). 2 pts for linking this acceptance/alignment directly to the elimination of irrational disturbance, leading to Ataraxia.%
\vspace{0.3cm}%
\item%
\textbf{S-2}%
 \ \ \textbf{Model Answer:} Practical Wisdom (*phronesis*) is the knowledge of what is truly good, bad, or indifferent, and the ability to navigate complex situations logically. Courage (*andreia*) is the ability to face challenges and difficulties with clarity and integrity. Courage relies entirely on Practical Wisdom because Wisdom determines *when* and *how* to act bravely. Mere physical bravery (recklessness) is insufficient because it lacks the rational judgment to discern if the risk is morally worthwhile or if the action is truly virtuous. Without Practical Wisdom, Courage is blind and can lead to unjust or intemperate actions; thus, true Stoic Courage is the rational willingness to act virtuously, even in the face of danger.%
 \\ \textit{Grading Notes:} 1 pt for defining Practical Wisdom as rational judgment/knowledge of good/bad. 1 pt for defining Courage as facing challenges with clarity/integrity. 3 pts for analyzing the interdependence: explaining that Wisdom directs Courage, ensuring the action is morally justified and not reckless, thereby making the virtue holistic.%
\vspace{0.3cm}%
\item%
\textbf{S-3}%
 \ \ \textbf{Model Answer:} Internal Factor (Within Control): The student's effort, preparation methods, attitude towards the result, and the judgment they form about the failing grade.
External Factor (Outside Control): The difficulty of the exam, the professor's grading curve, the actions of other students, or the final score assigned.
Stoic Response to the External Factor: The student must accept the external factor (the failing grade) as an indifferent event governed by the Logos. The appropriate response is not to lament the score itself, but to focus internally on adjusting their study methods (Practical Wisdom) and maintaining their inner composure (Temperance), recognizing that the grade is merely a preferred indifferent, not a true evil.%
 \\ \textit{Grading Notes:} 1 pt for correctly identifying an internal factor (e.g., effort, judgment, approach). 1 pt for correctly identifying an external factor (e.g., the grade, the professor's decision). 3 pts for explaining the appropriate Stoic response: acceptance of the external factor as indifferent/outside control, and redirecting focus back to internal virtuous action (e.g., improving methods, maintaining composure).%
\vspace{0.3cm}%
\item%
\textbf{S-4}%
 \ \ \textbf{Model Answer:} Marcus Aurelius, as Emperor, applied Justice on a massive, systemic scale. His focus was on administering the law fairly across the entire Roman Empire, making decisions that affected millions, and ensuring the common good during times of war and plague. His Justice was inherently political and judicial. Seneca, as an advisor and wealthy writer, focused Justice more on personal ethics and social responsibility. His application emphasized humane treatment (e.g., towards slaves, as noted in the source material) and the moral obligation of the privileged to act fairly toward the less fortunate. While both valued Justice, Aurelius focused on institutional/governance Justice, whereas Seneca focused on interpersonal/ethical Justice.%
 \\ \textit{Grading Notes:} 1 pt for identifying Marcus Aurelius' context (Emperor/Ruler) and Seneca's context (Advisor/Writer/Wealthy Citizen). 2 pts for explaining Aurelius' scope of Justice (systemic, judicial, common good, empire-wide). 2 pts for explaining Seneca's scope of Justice (interpersonal, ethical treatment, humane responsibility, specifically citing examples like treatment of slaves).%
\vspace{0.3cm}%
\item%
\textbf{S-5}%
 \ \ \textbf{Model Answer:} The Epictetan principle establishes that emotional suffering (disturbances) is not caused by external events (e.g., losing a job, receiving a bad grade), but by the internal, often irrational, judgments or beliefs we attach to those events. REBT, founded by Albert Ellis, adopts this premise directly through its A-B-C model. 'A' (Activating Event) does not directly cause 'C' (Consequence/Emotional Reaction); instead, 'B' (Beliefs/Judgments about A) causes C. Both Stoicism and REBT aim to identify and challenge these irrational beliefs (B) to achieve a rational, healthy emotional consequence (C), thereby mirroring the Stoic goal of replacing irrational passions with rational judgments to attain tranquility.%
 \\ \textit{Grading Notes:} 2 pts for clearly explaining the Stoic principle: that suffering is caused by internal judgments/beliefs, not external events. 3 pts for linking this directly to REBT: mentioning the A-B-C model (or equivalent explanation) and stating that REBT's core function is identifying and challenging the irrational 'B' (Beliefs), demonstrating the direct philosophical lineage.%
\vspace{0.3cm}%
\item%
\textbf{P-1}%
 \ \ \textbf{Model Answer:} The Stoic concept of Logos represents the fundamental, rational, and deterministic order governing the cosmos. Acceptance of Logos means recognizing that all external events are necessary and interconnected, and therefore, outside of one's control. The Cardinal Virtues constitute the sole good and the only sphere where humans possess true control: their judgments and intentions.

Acceptance of Logos is foundational because it dictates the proper scope of virtuous action. Courage, in the Stoic sense, is not recklessness but the willingness to face necessary hardship (as dictated by Logos) with integrity and clarity. If a Stoic believes an external event (e.g., illness, loss of status) is fundamentally irrational or unjust, they cannot face it courageously; they must first accept it as part of the rational whole. 

Similarly, Justice requires recognizing the shared rationality (Logos) inherent in all human beings, regardless of their social status or actions. By viewing others as fellow participants in the rational order, the Stoic is compelled to treat them fairly and humanely, even when they commit wrongs, understanding that their errors stem from ignorance of the Logos, not malice.%
 \\ \textit{Grading Notes:} Total 10 points.
1. Definition/Relationship (3 pts): 1 pt for defining Logos as rational order; 2 pts for explaining that Logos defines the external reality that Virtue must respond to (Virtue is the only good/control).
2. Logos and Courage (3 pts): Must explain that Courage relies on accepting the necessity of external events (Logos) to act rightly despite fear or pain.
3. Logos and Justice (3 pts): Must explain that Justice stems from recognizing the shared rationality (Logos) in all humanity, compelling humane treatment.
4. Precision/Clarity (1 pt): Use of precise terminology (e.g., 'indifferents,' 'necessity').%
\vspace{0.3cm}%
\item%
\textbf{P-2}%
 \ \ \textbf{Model Answer:} Analysis using the Dichotomy of Control:
1. **External Factors (Out of Control):** The colleague's plagiarism, the professional censure, and the loss of the grant are all external events. These are 'preferred indifferents' that, while undesirable, do not affect the Stoic's moral character.
2. **Internal Factors (In Control):** The Stoic controls their judgment (assent) regarding the event, their emotional reaction (e.g., anger, despair), and their subsequent intention/action.

The most crucial Cardinal Virtue is **Practical Wisdom** (Prudence). This virtue is the knowledge of what is truly good, bad, or indifferent. 

Application to maintain Ataraxia:
Practical Wisdom dictates that the only true harm is moral corruption. The loss of the grant and the censure are external harms, not moral harms. The Stoic applies Practical Wisdom by judging the situation correctly: the colleague acted wrongly, but my reaction must be virtuous. The Stoic focuses on controlling the internal narrative (e.g., avoiding the irrational judgment that 'I deserve better' or 'This is intolerable') and instead focuses on the next virtuous step, such as calmly addressing the situation through appropriate channels without succumbing to rage or vengeance.%
 \\ \textit{Grading Notes:} Total 10 points.
1. Dichotomy Application (3 pts): Correctly identifies the plagiarism/censure/loss as external factors/indifferents and the judgment/response as internal control.
2. Virtue Identification (3 pts): Identifies Practical Wisdom (or Temperance, if well-justified) as the key virtue. Must justify why this virtue is needed (e.g., Practical Wisdom is needed for correct judgment; Temperance is needed for controlling the impulse to rage).
3. Application to Ataraxia (4 pts): Explains the mechanism: the virtue is applied by correcting the internal judgment (removing the irrational belief that the external event is intolerable) and focusing only on virtuous response, thereby preventing emotional disturbance and achieving tranquility.%
\vspace{0.3cm}%
\item%
\textbf{P-3}%
 \ \ \textbf{Model Answer:} $\text{Ataraxia}$ is the Stoic goal of tranquility, freedom from disturbance, and inner peace. $\text{Logos}$ is the rational, providential, and deterministic structure of the universe—the interconnected web of cause and effect that governs all events.

Epictetus taught that 'Men are disturbed not by things, but by the views which they take of them.' Emotional disturbances (pathē, such as fear, desire, or distress) arise when we attach irrational judgments to external events, treating them as inherently good or bad, rather than indifferent.

The Stoic sage utilizes their understanding of $\text{Logos}$ to eliminate these disturbances by recognizing that all external events, including suffering, are necessary components of the rational whole. If an event is necessary according to $\text{Logos}$, it cannot logically be 'bad' in a moral sense. By accepting the event as part of the rational order, the sage removes the irrational judgment that the event 'should not be happening.' This removal of false judgment eliminates the root cause of the emotional disturbance, thereby allowing the sage to remain undisturbed and achieve $\text{Ataraxia}$.%
 \\ \textit{Grading Notes:} Total 10 points.
1. Definition of Ataraxia (2 pts): Accurate definition of tranquility/freedom from disturbance.
2. Definition of Logos (2 pts): Accurate definition of the rational/deterministic structure of the universe.
3. Synthesis (6 pts): Must connect all three elements. 2 pts for introducing Epictetus's core teaching (suffering from judgments, not events). 4 pts for explaining the mechanism: understanding Logos confirms the necessity of external events, which allows the sage to withdraw assent from the irrational judgment that the event is intolerable/unjust, thus eliminating the pathē and achieving Ataraxia.%
\vspace{0.3cm}%
\end{enumerate}

%
\end{document}